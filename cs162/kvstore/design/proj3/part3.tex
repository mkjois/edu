\section{KVCache}

\subsection{Overview}
KVCache stands for Key-Value Cache. It is a direct mapped cache that basically uses 
a hashtable of LinkedLists. Keys are strings and values are LinkedLists simulating 
sets. We make sure that our LinkedList does not contain duplicate entries. Replacement 
is done based on the second chance algorithm, all values have a "used" boolean value associated 
with them. Entries start out with a value of false and if they get accessed then we change the 
value to true. The most recently accessed value is at the tail of the queue and the least 
recently accessed value is at the head of the queue.
\subsection{Correctness Constraints}
\begin{itemize}
\item Every entry enters the queue with its used bit set to 0.
\item Whenever something is accessed, it's used bit is set to 1 and it is moved to the tail of the Linked List.
\item Two copies of the same element cannot exist in the cache (it is a set)
\end{itemize}

\subsection{Declarations}
public class \texttt{KVCache} implements KeyValueInterface
\begin{itemize}
\item Java.util.hashtable cache - our "cache"
\item int numSets
\item public KVCache(numSets, maxElemsPerSet) - constructor
\item String get(key) - retrieves an entry from the cache
\item void put(key, value) - puts a value in the cache 
\item void del(key) - removes an entry from the cache
\item Lock getLock(key) - gets a lock for a set corresponding to a given key
\item int getSetId(key) - get the id of a set that contains a specific key
\item String toXML() - serialoze this store to XML
\item void makeSpace - if a set is full, do second chance on it to make space
\end{itemize}

private class \texttt{lockList} - linked list with a lock, otherwise same as a normal LinkedList
\begin{itemize}
\item java ReentrantLock Lock
\item LinkedList<cacheNode> set
\item int size
\item boolean isFull() - check to see if this set is full
\item cacheNode find(key) - find given value for a key, if it isnt there, return null
\item LinkedList<cacheNode> getList() - returns this.set
\item ReentrantLock getLock() - returns this.lock
\end{itemize}

private class \texttt{cacheNode} - a node that has a used bit and a container for the entry
\begin{itemize}
\item boolean used
\item String key
\item String value 
\item getUsed()
\item setUsed(boolean b)
\item getKey()
\item getValue()
\item setValue(String v)
\end{itemize}

\subsection{Descriptions}
\begin{verbatim}
class KVCache
    Java.util.hashtable cache
    int numSets
    
    KVCache(int numSets, int maxElemsPerSet)
        cache = new hashTable<lockList>
        this.numSets = numSets
        for a in range(numsets)
            add a LockList of max size maxElemsPerSet to the HashTable
    
    get(key)
        int a = getSetId(key)
        cacheNode value = l.find(key)
        if value != null
            value.used = true
            return value.value
        return null
    
    put(key,value)
        int a = getSetId(key)
        Locklist l = cache.get(a)
        cacheNode v = l.find(key)
        if v != null
            v.value = value
            v.used = true
        else
            cacheNode n = new cacheNode(key, value)
            if l.isFull()
                makeSpace(l)
            l.add(n)
    
    del(key)
        int a = getSetId(key)
        Locklist l = cache.get(a)
        cacheNode value = l.find(key)
        if value != null
            l.remove(value)
    
    getLock(key)
        int a = getSetId(key)
        Locklist l = cache.get(a)
        return l.getLock()
    
    getSetId(key)
        int h = key.hashcode()
        int id = mod(h, this.numSets)
        f (id < 0 ) id = id + this.numSets
        return id
    
    makeSpace(lockedlist list)
        boolean replaced = false
        for node in list
            if node.used == false
                list.remove(node)
                replaced = true
            node.used = false
        if !replaced
            remove the head
        return 
    
    toXML()
       Element rootNode = new Element("KVCache");
    		
    	for (int s = 0; s < this.numSets; s++){
    		Element set = new Element("Set");
    		int setID = getSetId(String.valueOf(s));
    		String name = String.valueOf(setID);
    		
    		Attribute Id = new Attribute("Id", name);
    		set.addAttribute(Id);
    		
    		LockList l = cache.get(setID);
    		for (cacheNode c : l.set){
    			Element cacheEntry = new Element("CacheEntry");
    			Attribute isReferenced = new Attribute("isReferenced", String.valueOf(c.getUsed()));
    			cacheEntry.addAttribute(isReferenced);
    			
    			Element key = new Element("Key");
    			key.appendChild(c.getKey());
    			
    			Element value = new Element("Value");
    			value.appendChild(c.getValue());
    			
    			cacheEntry.appendChild(key);
    			cacheEntry.appendChild(value);
    			set.appendChild(cacheEntry);
    		}
    		
    		rootNode.appendChild(set);
    		
    	}
    	Document output = new Document(rootNode);
    	return output.toXML();
    	
    	
class LockList
   
    java.util.lock Lock
    LinkedList<cacheNode> set
    int size
   
    LockList(id, size)
        this.id = id
        this.size = size
        set = new LinkedList<cacheNode>
        this.Lock = new Lock
    
    isFull()
        return this.size >= this.set.size()
        
    find(key)
        for node in set
            if key == node.key
                return node
        return null
    
    getList()
        return this.set
    
    getLock()
        return this.lock
    
class CacheNode
    
    boolean used
    String key
    String value
    CacheNode(key, value)
        this.key = key
        this.value = value
        this.used = false
    
    getUsed()
        return this.used
    
    setUsed(boolean b)
        this.used = b
        
    getKey()
        return this.key
        
    getValue()
        return this.value
    
    setValue(String value)
        this.value = value
        
\end{verbatim}
\subsection{Testing Plan}
I made 5 tests, all are in KVCacheTest.java
    1.test filling up the cache fully
    2.make sure no duplicate elements in sets
    3.test dell()
    4.test makeSpace()
    5.test toXML()
