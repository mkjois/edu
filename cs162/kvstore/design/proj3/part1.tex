\section{KVClient}

\subsection{Overview}
\texttt{KVClient} issues requests and handles responses by interacting with \texttt{KVMessage} objects. Each \texttt{KVClient} is initialized with the server that responds to requests and the port it is listening on. To request a connection, it calls \texttt{connectHost()}, initializes a Socket to interact with the server and returns it. The methods to store, get, and remove key-value pairs on the server for \texttt{KVClient} are \texttt{put(), get(), del()} respectively. When \texttt{KVClient} is finished, it calls \texttt{closeHost()} on a best-effort to close the socket it had been using.

\subsection{Correctness Constraints}
\begin{itemize}
\item When KVClient puts a key-value pair, it will send it through to the KVServer connected to the other end of the socket if there is no error.
\item If KVClient gets a key that has previously been put, it will get back the value for that key.
\item If KVClient tries to get a key that has previously been deleted, it will not get back the value.
\end{itemize}

\subsection{Declarations}
\begin{itemize}
\item String server - The server the client is connecting to.
\item int port - The port on the server the client is connecting to.
\end{itemize}
\subsection{Descriptions}

\begin{verbatim}
class KVClient:
    def connectHost():
        mySock = new Socket(server, port)
        if error:
            throw a KVException with appropriate string constant
        return mySock

    def closeHost(Socket sock):
        sock.close()
        if error:
            Do nothing.

    def put(key, value):
        newSock = connectHost()
        message = new KVMessage("put")
        message.setKey(key)
        message.setValue(value)
        message.sendMessage(newSock)
        returnMessage = new KVMessage(newSock)
        retVal = returnMessage.getMessage()
        if no error:
            closeHost(newSock)
        else:
            throw KVException with appropriate string constant

    def get(key):
        newSock = connectHost()
        message = new KVMessage("get")
        message.setKey(key)
        message.sendMessage(newSock)
        returnMessage = new KVMessage(newSock)
        if no error:
            retVal = returnMessage.getValue()
            closeHost(newSock)
            return retVal
        else:
            throw KVException with appropriate string constant

    def del(key):
        newSock = connectHost()
        message = new KVMessage("del")
        message.setKey(key)
        message.sendMessage(newSock)
        returnMessage = new KVMessage(newSock)
        retVal = returnMessage.getMessage()
        if no error:
            closeHost(newSock)
        else:
            throw KVException with appropriate string constant

\end{verbatim}

\subsection{Testing Plan}
We plan to test task1 with the testing framework and modifying from SampleClient and SampleServer to cover different scenarios.
\begin{itemize}
\item Basic functionality
\subitem Put unique key and value pair
\subitem Get it back
\subitem Check if they are equal
\item Test error
\subitem Put empty string
\subitem Check for correct error
\end{itemize}