
\documentclass{article}
\usepackage[utf8]{inputenc}
\usepackage{listings}
\usepackage{authblk}
\usepackage{enumerate}
\usepackage[usenames,dvipsnames]{color}
\usepackage[top=1in, bottom=1in, left=1in, right=1in]{geometry}

\title{Project Phase 2: Multiprogramming}
\author{Romil Singapuri (ha), Alexander Wang (is), Manohar Jois (jp),
\\Shrayus Gupta (jq), Junseok Lee (fy)}
\date{March 21, 2014}
\begin{document}

\maketitle

\section{System calls: creat(), open(), read(), write(), close(), unlink()}
\subsection{Overview}
These six system functions allow the user to access files and streams in their programs safely and without risking failure of the operating system. No syscall, combination of syscalls, or combination of arguments to syscalls should be able to corrupt or crash the kernel, except the \textit{root} (first) process possibly calling \texttt{halt()}. We deal only with virtual addresses, so we must make use of the methods to read/write virtual memory to interface with the underlying file system. Each process references a file/stream using a recyclable integer called a \textit{file descriptor} from 0-15, and separate sets of file descriptors are given to each process.
\subsection{Correctness Constraints}
\begin{itemize}
\item When initialized, file descriptors 0 and 1 for all processes refer to standard input and output, respectively. A user can close them at any time, but may not reopen them.
\item \texttt{halt()} can only execute for the root process, otherwise return -1 immediately.
\item Correctness constraints for the system calls:
\begin{itemize}
\item \texttt{creat()} -- Open a disk file (never a stream) by name, creating it if no such file exists, and return the file descriptor. If called multiple times, simply open the file. If user process already has 16 valid file descriptors, or file system couldn't open the file, return -1.
\item \texttt{open()} -- Open a disk file (never a stream) by name. If called multiple times, return a new file descriptor for each call. If file system couldn't open the file, return -1.
\item \texttt{read()} -- Read bytes from an open file/stream and return the number of bytes immediately read, advancing the file pointer position by this number. If the file descriptor is invalid for this process, reading from file fatally errs, or writing to virtual memory fatally errs, return -1. 
\item \texttt{write()} -- Write bytes to an open file/stream and return the number of bytes successfully written, advancing the file pointer position by this number. If less than the number of bytes requested to be written, this is an error the user must deal with. If the file descriptor is invalid, writing to the file fatally errs, or reading from virtual memory fatally errs, return -1.
\item \texttt{close()} -- Invalidate a file descriptor, flush any data written to the associated file/stream, and release its resources. If an invalid file descriptor or other error, return -1.
\item \texttt{unlink()} -- Signal the internal file system to delete the file, while also invalidating the file descriptor for this process. Other processes also cannot use this file except to call close() on its file descriptor. If an error occured within the file system, return -1.
\end{itemize}
\item Any combination of syscalls (valid 0-9 or invalid) and/or arguments to the syscalls must result in the correct behavior as specified above with a valid return value, a -1 for an error, or the termination of the process and the deallocation of its resources.
\end{itemize}
\subsection{Declarations}
\large
***No significant changes.***\\\\
\normalsize
\textbf{UserProcess:}
\begin{itemize}
\item \texttt{OpenFile[] files} -- Set of open files for this process, indexed by file descriptor. Invalid file descriptors correspond to null entries.
\item \texttt{LinkedList<int> descriptors} -- Queue of available file descriptors. Dequeue to get one for a new file in creat() and enqueue an invalidated one in close().
\item \texttt{boolean bogusArguments(pointer, fileDesc, count)} -- Returns true if pointer < 0, fileDesc not in [0,15], or count < 0
\item Syscalls will be handled internally using \texttt{handle<Syscall>()} methods.
\item \texttt{void kill()} -- Defined in section 3.
\end{itemize}
\subsection{Descriptions}
\large
***No significant changes.***
\normalsize
\begin{verbatim}
class UserProcess:
    def UserProcess():
        // in addition to current constructor code
        files = new OpenFile[16]
        descriptors = empty Java linked list of integers
        enqueue 2-15 into descriptors, in that order
        assign files[0] to UserKernel.console.openForReading()
        assign files[1] to UserKernel.console.openForWriting()

    def handleSyscall(syscall, a0, a1, a2, a3):
        switch syscall:
            call handlers based on syscall with arguments in same order
            default:
                kill() this process

    def handleCreat(nameAddr):
        return handleCreateOrOpen(nameAddr, true)

    def handleOpen(nameAddr):
        return handleCreateOrOpen(nameAddr, false)

    def handleCreateOrOpen(nameAddr, create): // create is a boolean
        if bogusArguments(nameAddr, 0, 0): return -1
        if descriptors is empty: return -1
        name = readVirtualMemoryString(nameAddr, 256)
        if name is null: return -1
        newFile = ThreadedKernel.fileSystem.open(name, create)
        if newFile is null: return -1
        newDescriptor = descriptors.dequeue()
        assert files[newDescriptor] is null
        files[newDescriptor] = newFile
        return newDescriptor

    def handleRead(fileDesc, buffAddr, count):
        if bogusArguments(buffAddr, fileDesc, count): return -1
        if files[fileDesc] is null: return -1
        buffer = new byte[count]
        bytesReadFromFile = files[fileDesc].read(buffer, 0, count)
        if bytesReadFromFile == -1: return -1
        bytesWrittenToMem = writeVirtualMemory(buffAddr, buffer, 0, bytesReadFromFile)
        if bytesWrittenToMem < bytesReadFromFile: return -1
        else return bytesReadFromFile
    
    def handleWrite(fileDesc, buffAddr, count):
        if bogusArguments(buffAddr, fileDesc, count): return -1
        if files[fileDesc] is null: return -1
        buffer = new byte[count]
        bytesReadFromMem = readVirtualMemory(buffAddr, buffer)
        if bytesReadFromMem < count: return -1
        bytesWrittenToFile = files[fileDesc].write(buffer, 0, count)
        if bytesWrittenToFile == -1: return -1
        else return bytesWrittenToFile

    def handleClose(fileDesc):
        if bogusArguments(0, fileDesc, 0): return -1
        if files[fileDesc] is null: return -1
        files[fileDesc].close()
        set files[fileDesc] to null
        enqueue fileDesc into descriptors
        return 0

    def handleUnlink(nameAddr):
        if bogusArguments(nameAddr, 0, 0): return -1
        name = readVirtualMemoryString(nameAddr, 256)
        if name is null: return -1
        if ThreadedKernel.fileSystem.remove(name) fails: return -1
        else return 0

    def handleHalt():
        if current process is root: Machine.halt()
        else return -1

    def bogusArguments(pointer, fileDesc, count):
        return (pointer < 0 or count < 0 or
                fileDesc < 0 or fileDesc > 15)
\end{verbatim}
\subsection{Testing Plan}
We used 5 test programs written in C and compiled to MIPS using a cross compiler. These tests are mostly of the glass box variety while also testing to bulletproof the kernel against any user input, whether unintentionally bad or intentionally malicious. All tests are in the \texttt{test} directory.
\begin{itemize}
\item \texttt{p1t1.c} tests... \begin{itemize}
\item correct creat() functionality.
\item open() on nonexistent file.
\item reading from stdin.
\item writing to a file.
\end{itemize}
\item \texttt{p1t2.c} tests... \begin{itemize}
\item correct open() functionality.
\item reading from a file.
\item writing to stdout.
\end{itemize}
\item \texttt{p1t3.c} tests... \begin{itemize}
\item reading from one file and writing read data to another.
\item correct close() functionality.
\item correct unlink() functionality.
\item calling close() after unlink().
\end{itemize}
\item \texttt{bullets1.c} tests... \begin{itemize}
\item creat() on a file name $<=$ 256 characters.
\item creat() on a file name of 257 characters.
\item creat() on an invalid (negative) pointer.
\item if creat() returns file descriptors up to 15.
\item that successive calls to creat() return -1.
\item close() on invalid file descriptors.
\item unlink() on nonexistent files.
\item unlink() on a file name $<=$ 256 characters.
\item unlink() on a file name of 257 characters.
\item unlink() on an invalid (negative) pointer.
\end{itemize}
\item \texttt{bullets2.c} tests... \begin{itemize}
\item reading more from stdin than given buffer can hold.
\item writing more to stdout than given buffer contained.
\item reading from stdout.
\item writing to stdin.
\item reading negative bytes from a file.
\item writing negative bytes to a file.
\item reading to a nonexistent file.
\item writing to a nonexistent file.
\item reading from a file and writing data to another file in blocks, using a while loop.
\item reading from a closed file.
\item writing to closed file.
\item open() on a file already opened.
\item open() on an unlinked file.
\item creat() on an unlinked file.
\item reading second version of file using a file descriptor for the first version.
\item reading with an invalid file descriptor.
\item writing with an invalid file descriptor.
\end{itemize}
\end{itemize}

\section{Support for Multiprogramming}

\subsection{Overview}
To implement support for multiprogramming, we need to keep a record of pages available for a userProcess in the form of a global linked list in the UserKernal. \\\\
When a UserProcess initializes and calls loadSections(), we know both exactly the size and number of pages needed to store the process in memory given that these programs have no dynamic memory allocation capability. The UserProcess is given a local pageTable that maps each virtual page to physical page, unless the UserProcess's memory needs exceed that of available memory, in which case exec() will be called. If exec() was not called, then the UserKernal will update its global list of space to indicate the parts of memory allocated and call setPageTable, completing the mapping of virtual memory to physical memory.\\\\
When the UserProcess calls readVirtualMemory or writeVirtualMemory, it will access its pageTable, see whether or not the operation is allowed in the TranslationEntry corresponding to the virtual page table  number, and then perform the operation or return.

\subsection{Correctness Constraints}
\begin{itemize}
\item A process's memory is freed upon exit and empty upon intialize
\item No two processes will collide/overlap in physical memory.
\item An initialized UserProcess will always have the physical address spaces given to it in its pageTable.
\item A page in physical memory will have enough space to store $2^{o}$ bytes, where o=offset.
\item Physical memory will be used to its fullest by allocating pages in gaps if available.
\item When a process is killed or exits in any way, the pages allocated to it will be freed
\item For every page written and read in a process, the TranslationEntry will reflect that with its used/dirty bits set.
\end{itemize}

\subsection{Declarations}
\texttt{LinkedList freePages} - Global list of free pages in physical memory available for processes to allocate. \\
\texttt{Lock pageLock} - Global static lock for when user processes manage the freePages LinkedList

\subsection{Descriptions}

\begin{verbatim}
class UserKernal:
    def initialize():
        //Add the following
        Lock pageLock = new Lock()
        freePages = LinkedList()
        for i in Machine.processor().getNumPhysPages():
            freePages.add((i, Machine.processor().makeAddress(i, 0))
            
    def getFreePages():
        return freePages
        
class UserProcess:
    def UserProcess():
        loadSections()
    
    def loadSection():
        //In addition to the insufficient memory check

        // load sections
        ArrayList coffPageMappings, coffVPNs, readOnlyPages
        UserKernel.pageLock.acquire()
        freePages = UserKernel.getFreePages()
        if freePages.size() < numPages:
            don't load and exit
        for i in range(coff.getNumSections()):
            Coffsection section = coff.getSection(i)
            sectionReadOnly = section.isReadOnly()
            for j in range(section.getLength()):
                vpn = section.getFirstVPN()+j
                ppn = freePages.pop()
                coffMapping = (vpn, ppn)
                coffPageMappings.add(coffMapping)
                coffVPNs.add(vpn)
                if sectionReadOnly:
                    readOnly.add(ppn)
                section.loadPage(j, ppn)
    
    def writeVirtualMemory(vaddr, data, offset, length):
        //Along with assertion check
        currentAmount = 0
        for i in range(length):
            vpn = Machine.processor().pageFromAddress(vaddr+i)
            pageOffset = Machine.processor().offsetFromAddress(vaddr+i)
            if vpn >= pageTable.length:
                return currentAmount //invalid
            mapping = pageTable[vpn]
            if mapping not valid:
                return currentAmount
            physAddr = Machine.processor().makeAddress(mapping.ppn, pageOffset)
            if physAddr > physMemory.length:
                return currentAmount
            if mapping not readOnly:
                copy data, set ued and dirty bits true
                currentAmount++
            else:
                return currentAmount;

    def readVirtualMemory(vaddr, data, offset, length):
        //Along with assertion check
        currentAmount = 0
        for i in range(length):
            vpn = Machine.processor().pageFromAddress(vaddr+i)
            pageOffset = Machine.processor().offsetFromAddress(vaddr+i)
            if vpn >= pageTable.length:
                return currentAmount //invalid
            mapping = pageTable[vpn]
            if mapping not valid:
                return currentAmount
            physAddr = Machine.processor().makeAddress(mapping.ppn, pageOffset)
            if physAddr > physMemory.length:
                return currentAmount
            copy data, set used bits true
            currentAmount++
    
    def unloadSections():
        pageLock.acquire()
        get free pages from Kernel
        for i in range(numPages):
            freePages.add(ppn's in page table entry)
        pageLock.release()
    
\end{verbatim}

\subsection{Testing Plan}
We plan to test Multiprogramming by using the shell.
\begin{itemize}
\item Exit out of console
\subitem Spawning the shell uses multiprogramming so if we can press q and spawn a shell out of the console, the basic parts of it work.
\item Write tests in c files
\subitem Included joinTest to spawn more than one processes and if it returns correctly, we know it can handle up to its maximum allowed processes
\item Run multiple processes in background
\subitem Run a program and append a amphersand to make it a background process. This way we can run multiple copies of matmult and other tests designed to strain the processor.
\end{itemize}

\section{System calls: {\ttfamily exec()}, {\ttfamily join()}, and {\ttfamily exit()}}
\subsection{Overview}
Our goal is to implement {\ttfamily exec()}, {\ttfamily join()}, and {\ttfamily exit()} syscalls such that they work correctly and are ``bulletproof.'' These perform the functions of executing child processes, suspending execution until child processes complete, and ending processes, respectively. Full descriptions are documented in {\ttfamily syscall.h}.
\subsection{Correctness Constraints}
\begin{itemize}
\item No syscall argument shall cause a kernel crash.
\item No syscall argument shall corrupt a kernel data structure.
\item No allocated memory shall ever belong to a terminated process.
\item If a process calls {\ttfamily join()} with its child as an argument, it shall be suspended until the joined process exits, at which point it will return 0 if the child exited abnormally and 1 otherwise.
\item If a process calls {\ttfamily join()} on a process that is not its child, it shall not be suspended, and {\ttfamily join()} shall immediately return $-1$.
\item If a process calls {\ttfamily exec()} with a valid executable, that executable shall run in a new process.
\item If a process calls {\ttfamily exec()} with an invalid executable, {\ttfamily exec()} shall immediately return $-1$.
\item If a process calls {\ttfamily exit()}, it shall immediately be terminated.
\item Calls to {\ttfamily exit()} do not return.
\item If a process is joined, exit status shall be written to the appropriate location in the parent process's virtual memory space.
\item No two processes shall have the same PID.
\item The root process shall have PID 0.

\end{itemize}
\subsection{Declarations}
\textbf{UserProcess:}
\begin{itemize}
\item{\ttfamily private static final Lock counterLock} - A lock for {\ttfamily counter}.
\item{\ttfamily private static final Lock activeLock} - A lock for {\ttfamily active}.
\item{\ttfamily private static final Lock processesLock} - A lock for {\ttfamily processes}.
\item{\ttfamily private static final Lock statusLock} - A lock for {\ttfamily statuses}
\item{\ttfamily private static int counter} - A counter for how many processes have been executed. Initialized to 0.
\item{\ttfamily private static int active} - The number of active processes.
\item{\ttfamily private static final HashMap<Integer, UserProcess> processes} - A mapping from PID to process.
\item{\ttfamily private static final HashMap<Integer, Integer> statuses} - A mapping from PID to exit status.
\item{\ttfamily private HashSet children} - Gives us access to all the children of this process. Allows us to verify in constant time that a process this process is trying to join() is indeed a child of this process. Keys are pid fields.
\item{\ttfamily private Lock childLock} - A lock for {\ttfamily children}.
\item{\ttfamily private int pid} - A globally unique process identifier (PID).
\item{\ttfamily private boolean exitedNormally} - Whether this process exited normally.
\item{\ttfamily private UThread thread} - The thread in which this process runs.
\item{\ttfamily private void kill()} - Handles abnormal termination.
\item{\ttfamily private void handleJoin(int pid, int status)} - Handles a {\ttfamily join()} syscall. Note that the status argument is a pointer to a location in this process's virtual memory space into which the child's exit status will be written.
\item{\ttfamily private void handleExec(int name, int argc, int argv)} - Handles an {\ttfamily exec()} syscall. The arguments are the same integers passed into {\ttfamily handleSyscall()}.
\item{\ttfamily private void handleExit(int status, boolean normal)} - Handles process termination, both normal and abnormal (specify in {\ttfamily normal} parameter), with exit status {\ttfamily status}.
\end{itemize}
\subsection{Descriptions}
\begin{verbatim}
/**
 * Handles system calls. See the Nachos source for full documentation.
 * Ellipses below indicate switch cases not relevant to this section.
 */
handleSyscall(syscall, a0, a1, a2, a3):
    switch (syscall):
        ...
        case syscallExit:
            terminate(a0, true)
        case syscallExec:
            return handleExec(a0, a1, a2)
        case syscallJoin:
            return handleJoin(a0, a1)
        ...

// Moved tracking of active processes and PID assignment into load()
handleExec(name, argc, argv):
    proc = new user process
    nameString = proc->read string from name
    argPointers = empty int array of length argc
    load argc words from address argv into argPointers
    args = empty string array of length argc
    load strings from addresses in argPointers into args
    success = proc->execute with name, args
    if success:
        acquire childLock
        add pid to children
        release childLock
        acquire processesLock
        add proc to processes with key pid
        release processesLock
        return proc->pid
    else:
        return -1

// Used static exit status HashMap to allow access to exit statuses
// even when join() is called on a process that has already terminated.
handleJoin(pid, status)
    acquire childLock
    if pid not in children:
        release childLock
        return -1
    release childLock
    acquire processesLock
    joining = processes->get(pid)
    release processesLock
    t = joining->thread
    if t is not null:
        t->join
    acquire statusLock
    s = statuses->get(pid)
    release statusLock
    write s to virtual memory location status
    if joining->exitedNormally:
        return 0
    return 0

execute(name, args):
    if not load(name, args): return false
    thread = new UThread
    thread->setName(name)
    thread->fork
    return true

kill():
    terminate(-1, false)

// Writes exit status to HashMap instead of directly into parent
// memory, allowing join() to work even when the process has already
// terminated.
handleExit(status, normal):
    exitedNormally = normal
    unloadSections()
    acquire activeLock
    decrement active
    if active == 0:
        kernel->terminate
    release activeLock
    acquire statusLock
    add status to statuses with key pid
    release statusLock
    set thread to null // helps with garbage collection
    close all files
    finish

// Added active process tracking and PID assignment into load().
load(name, args):
    perform load as per section 2
    if loadSections() succeeds:
        acquire activeLock
        increment active
        release activeLock
        acquire counterLock
        pid = counter
        increment counter
        release counterLock
    return value as per section 2
\end{verbatim}
\subsection{Testing Plan}
These tests involve programs written in C, which will be compiled by a MIPS cross-compiler.
\begin{itemize}
    \item Execute a process that executes two other processes. The parent process will join each child process in order. One child will join the other and immediately exit with its exit status being the return value of {\ttfamily join()}. The call to {\ttfamily join()} should fail, causing the child process to exit with status -1. The parent process reads the child process's exit status, and the test succeeds if and only if the value read by the parent process is -1. This test makes sure that a {\ttfamily join()} call to process that is not a child of the calling process fails and returns -1. It also checks that a parent process can read a child process's exit status correctly through {\ttfamily join()}.
    \item We have a C program that, when executed with no arguments, will execute itself with argument ``a''. Each subsequent instance of the process will execute itself with its argument incremented by 1. When the argument is ``d'', the process will not call {\ttfamily exec()}, but will instead exit with status 1. If the argument exists but is not ``d'', the process will join the process it has executed and exit with the child process's status plus 1. The initial process (the one executed with no arguments) joins the process it has executed and checks that its exit status is 3, in which case the test passes. This test ensures that {\ttfamily join()} works correctly in succession and that parent processes can correctly access their child processes' exit statuses.
\end{itemize}

\section{Lottery Scheduler}

\subsection{Overview}
The \texttt{Lottery Scheduler} extends the Priority Scheduler Class. It assigns every thread a number of "tickets" which determine the probability that it will get picked to have a resource. Shorter processes have higher priority. We solve the \texttt{Priority Inversion} problem by giving the running thread the summation of the tickets of all of the threads in queue for this resource.

\subsection{Correctness Constraints}

\begin{itemize}

\item The running thread has to have number of tickets equal to the summation (tickets of threads in waiting queue).
\item Tickets are not to be treated as objects, because we could potentially have billions of tickets in circulation at a time.
\item The number of tickets a process has is effectively the processes' priority. This means the the minimum priority if a process is 1 and the maximum priority is Integer.maxValue().
\item The total number of tickets in the system can not exceed Integer.maxValue().
\end{itemize}

\subsection{Declarations}

\textbf{LotteryScheduler extends PriorityScheduler:}\begin{itemize}

\item {\ttfamily priorityMinimum = 1}
\item {\ttfamily priorityMaximum = Integer.maxValue}
\item {\ttfamily newThreadQueue(transferPriority)} - Allocate a new lottery queue.
\item {\ttfamily getThreadState(thread)} - Return thread's scheduling state
\end{itemize}

\noindent\textbf{LotteryQueue extends PriorityQueue:} \begin{itemize}
\item {\ttfamily LThreadState holder} - Current holder. of this lock
\item {\ttfamily LinkedList<LThreadState> realQueue} - LinkedList holding all threads in line for this lock, I changed it from a priority queue, because there is no reason to compare threads now.
\item {\ttfamily LotteryQueue(transferPriority)} - calls super() then sets realQueue to a linked List instead of a Priority Queue.
\item {\ttfamily void waitForAccess(thread)} - puts a threads on this resource's wait queue.
\item {\ttfamily void acquire(thread)} - gives a thread the lock.
\item {\ttfamily KTread nextThread()} - returns the next thread what will get this resource and gives that thread the lock. Changed so that it works with LinkedLists and calls the correct functions in LThreadState.
\item {\ttfamily LThreadState pickNextThread()} - called by nextThread to choose the next thread that will get this lock. Changed so that it implements a lottery algorithm.
\item {\ttfamily int numTickets()} - Counts up the number of tickets allocated for this resource.
\end{itemize}

\noindent\textbf{LThreadState extends ThreadState:} \begin{itemize}
\item {\ttfamily LinkedList heldQueues}               - All the resources that this thread is holding.
\item {\ttfamily LotteryQueue waitQueue}              - A queue that this thread is waiting in.
\item {\ttfamily LThreadState(p)}                      
\item {\ttfamily void waitForAccess(waitQueue)}       - updates waitQueue.
\item {\ttfamily void release(heldQueue)}              - releases a lock and fixes priority for all threads affected.
\item {\ttfamily acquire(waitQueue)}                  - empties waitQueue and gives thread a lock.
\item {\ttfamily notifyPriorityChange()}              - called whenever priority of a thread is changed, updates priority for all threads connected to this one.
\item {\ttfamily recalculate()}                       - recalculates priority for current thread.
\end{itemize}

\subsection{Description}
\begin{verbatim}
class LotteryScheduler extends PriorityScheduler:
    int priorityMinimim
    int priorityMaximum
    
    newThreadQueue(transferPriority):
        return new LotteryQueue(transferPriority)
    
    
    getThreadState(KThread thread):
        return thread.schedulingState
    
    
    class LotteryQueue extends PriotityQueue:
        LThreadState holder
        LinkedList realQueue
        
        LotteryQueue(transferPriority):
            super()
            realQueue = LinkedList<LThreadState>()
        
        
        void waitForAccess(KThread thread):
            if transferPriority:
                holder.priority += thread.priority          
            threads.append(thread)                              
            thread.waitForAccess(this)                          
            
        
        LThreadState nextThread():
            oldHolder = holder                      
            oldHolder.priority = oldHolder.recalculate()
            holder = pickNextThread()
            remove new holder from queue
            holder.recalculate()
            notifyPriorityChange()
        
        void acquire():
            this.holder = getThreadState(thread)                
            this.holder.acquire(this)                           

        
        LThreadState pickNextThread():
            selection = Math.random from 1 to numTickets        
            for thread in threads:                               
                selection -= thread.priority                    
                if selection <= 0:                              
                    return thread           
        
        int numTickets():
            for all threads in queue:
                count up the number of tickets 
                    
        class LThreadState extends ThreadState:
            LinkedList heldQueue
            LotteryQueue waitQueue
            
            LThreadState(p):
                super()
                heldQueue = LinkedList
                
            waitForAccess(queue):
                this.waitQueue = queue
            
            acquire(queue):                                     
                this.waitQueue = null
                this.heldQueue.append(queue)
            
            release(heldQueue):                      
                heldQueue.remove(queue)
                this.priority = recalculate()
                notifyPriorityChange()
                
            notifyPriorityChange(value):
                if (waitQueue != null && waitQueue.holder != null && waitQueu.transferPriority):
            	 int a = waitQueue.holder.recalculate()
            	 waitQueue.holder.effPriority = a
            	 waitQueue.holder.notifyPriorityChange();
            
            int recalculate():
                for q in heldQueues:
                    effPriority += numTickets()
             
            
\end{verbatim}
\subsection{Testing Plan}
I will reuse many of the tests from proj1 PrioritySchduler.

\texttt{donationTest}
\begin{itemize}
\item R1 - holder(thread1 - 11), in queue: (thread2 - 2), (thread3 - 3)
\item priority of T1 - 6, T2 - 2, T3 - 3

\texttt{cascadeTest}
\item R1 - holder(T1 - 1), in queue (T2)
\item R2 - holder(T2 - 2), in queue (T3)
\item R3 - holder(T3 - 3)
\item T1's priority should be 6, T2's priority should be 5.
\item R2.nextThread(), now T1, T2, and T3's priority should be 3.
\end{itemize}

\end{document}