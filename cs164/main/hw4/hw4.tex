\documentclass[11pt]{article}
\usepackage{textcomp,geometry,graphicx,verbatim}
\usepackage{fancyhdr}
\usepackage{amsmath,amssymb,enumerate}
\pagestyle{fancy}
\def\Name{Manohar Jois}
\def\Homework{4} % Homework number - make sure to change for every homework!
\def\Session{Fall 2014}

% Extra commands
\let\origleft\left
\let\origright\right
\renewcommand{\left}{\mathopen{}\mathclose\bgroup\origleft}
\renewcommand{\right}{\aftergroup\egroup\origright}
\newcommand{\N}{\mathbb{N}}
\newcommand{\Z}{\mathbb{Z}}
\newcommand{\R}{\mathbb{R}}
\newcommand{\Q}{\mathbb{Q}}
\newcommand{\C}{\mathbb{C}}
\newcommand{\p}[1]{\left(#1\right)}
\renewcommand{\gcd}[1]{\text{gcd}\p{#1}}
\renewcommand{\deg}[1]{\text{deg}\p{#1}}
\renewcommand{\log}[1]{\text{log}\p{#1}}
\renewcommand{\ln}[1]{\text{ln}\p{#1}}
\newcommand{\logb}[2]{\text{log}_{#1}\p{#2}}
\newcommand{\BigOh}[1]{O\p{#1}}
\newcommand{\BigOmega}[1]{\Omega\p{#1}}
\newcommand{\BigTheta}[1]{\Theta\p{#1}}

\title{CS164\ \Session\  --- Answers to Homework \Homework}
\author{\Name}
\lhead{CS164\ \Session\  Homework \Homework\ Problem \theproblemnumber,\ \Name}

\begin{document}
\maketitle
\newcounter{problemnumber}
\setcounter{problemnumber}{0}

\section*{Exercise 1}
\stepcounter{problemnumber}
\begin{enumerate}[(P1)]
\item \begin{verbatim}
function fibCo(a, b)
    coroutine.yield(a)
    coroutine.yield(b)
    fibCo(a+b, a + 2*b)
end
\end{verbatim}
\item \begin{verbatim}
function takeCo(stream, n)
    while n > 0 do
        coroutine.yield(stream())
        n = n-1
    end
end
\end{verbatim}
\item \begin{verbatim}
function filterCo(stream, prop)
    while true do
        local val = stream()
        while not prop(val) do
            val = stream()
        end
        coroutine.yield(val)
    end
end
\end{verbatim}
\item \begin{verbatim}
function mapCo(stream, f)
    while true do
        coroutine.yield(f(stream()))
    end
end
\end{verbatim}
\end{enumerate}


\newpage
\section*{Exercise 2}
\stepcounter{problemnumber}
\begin{enumerate}[(P1)]
\item After the call to \textbf{f} returns, (d) the call to \textbf{envBind} executes.\\
(b) The JavaScript call stack is maintained so the interpreter knows to call \textbf{envBind} once all three of its arguments are evaluated.
\item \begin{verbatim}
1
2
\end{verbatim}
\item Bytecode consists of a flat, sequential instruction space where each instruction can be referenced by a program counter. The coroutine \textbf{f} can maintain its own program counter in its state as it executes and reference where it left off when it is resumed.
\item The interpreter should throw an error indicating that we can't resume into a coroutine that has completed execution. So we should track the state of each coroutine to be in one of three states: Running, Suspended or Terminated.
\item The interpreter should throw an error indicating there is no other coroutine to yield to. An easy way to track this is to maintain a pointer to a coroutine's parent, setting the field upon resumption of a coroutine. All \textbf{yield} statements will pass control and data back to a coroutine's parent.
\item \begin{verbatim}
data type Coroutine {
    Coroutine parent;
    int state;        // constants: {SUSPENDED, RUNNING, TERMINATED}
    int pc;           // program counter
    List body;        // list of sequential bytecode instructions
    Stack callStack;
    Dict env;         // bindings at the top of the call stack
}
\end{verbatim}
\item \textbf{Print statement at line 3:}\\
Stack frames(top down): \begin{itemize}
\item \texttt{feedCo} - line 29
\end{itemize}
Coroutines: \begin{verbatim}
MAIN {
    parent: null
    state: RUNNING
    pc: 3            // we'll go by line numbers instead of bytecode
    body: [...]      // lines 1-33
    callStack: {
        feedCo       // line 29
    }
    env: {
        feedCo: Lambda
        process: Lambda
        sourceCo: Lambda
    }
}

sourceCoFun {
    parent: null
    state: SUSPENDED
    pc: 22
    body: [...]      // lines 22-24
    callStack: {}
    env: {}
}
\end{verbatim}
Output: \texttt{Creating source coroutine.}\\\\
\textbf{Print statement at line 10:}\\
Stack frames (top down): \begin{itemize}
\item \texttt{resume} - line 30
\item \texttt{processCoFun} - line 8
\end{itemize}
Coroutines: \begin{verbatim}
MAIN {
    parent: null
    state: SUSPENDED
    pc: 30           // we'll go by line numbers instead of bytecode
    body: [...]      // lines 1-33
    callStack: {
        resume       // line 30
    }
    env: {
        feedCo: Lambda
        process: Lambda
        sourceCo: Lambda
        fc: Coroutine
    }
}

processCoFun {
    parent: MAIN
    state: RUNNING
    pc: 10
    body: [...]      // lines 9-15
    callStack: {}
    env: {
        sc: Coroutine
        unused: null
        x1: 1
    }
}

sourceCoFun {
    parent: processCoFun
    state: SUSPENDED
    pc: 22
    body: [...]      // lines 22-24
    callStack: {
        yield        // line 22
    }
    env: {
        unused: null
    }
}
\end{verbatim}
Output: \texttt{Running process coroutine.}\\\\
\textbf{Print statement at line 23:}\\
Stack frames (top down): \begin{itemize}
\item \texttt{resume} - line 31
\item \texttt{processCoFun} - line 8
\item \texttt{resume} - line 12
\end{itemize}
Coroutines: \begin{verbatim}
MAIN {
    parent: null
    state: SUSPENDED
    pc: 31           // we'll go by line numbers instead of bytecode
    body: [...]      // lines 1-33
    callStack: {
        resume       // line 31
    }
    env: {
        feedCo: Lambda
        process: Lambda
        sourceCo: Lambda
        fc: Coroutine
        r1: 1
    }
}

processCoFun {
    parent: MAIN
    state: SUSPENDED
    pc: 12
    body: [...]      // lines 9-15
    callStack: {
        resume       // line 23
    }
    env: {
        sc: Coroutine
        unused: null
        x1: 1
    }
}

sourceCoFun {
    parent: processCoFun
    state: RUNNING
    pc: 23
    body: [...]      // lines 22-24
    callStack: {}
    env: {
        unused: null
    }
}
\end{verbatim}
Output: \texttt{Running source coroutine.}\\\\
\textbf{Print statement at line 15:}\\
Stack frames (top down): \begin{itemize}
\item \texttt{resume} - line 32
\item \texttt{processCoFun} - line 8
\end{itemize}
Coroutines: \begin{verbatim}
MAIN {
    parent: null
    state: SUSPENDED
    pc: 32           // we'll go by line numbers instead of bytecode
    body: [...]      // lines 1-33
    callStack: {
        resume       // line 32
    }
    env: {
        feedCo: Lambda
        process: Lambda
        sourceCo: Lambda
        fc: Coroutine
        r1: 1
        r2: 2
    }
}

processCoFun {
    parent: MAIN
    state: RUNNING
    pc: 15
    body: [...]      // lines 9-15
    callStack: {}
    env: {
        sc: Coroutine
        unused: null
        x1: 1
        x2: 2
        x3: null
    }
}

sourceCoFun {
    parent: processCoFun
    state: TERMINATED
    pc: 25
    body: [...]      // lines 22-24
    callStack: {}
    env: {
        unused: null
    }
}
\end{verbatim}
Output: \texttt{Finishing process coroutine.}\\\\
\textbf{Print statement at line 33:}\\
Stack frames (top down): None\\\\
Coroutines: \begin{verbatim}
MAIN {
    parent: null
    state: RUNNING
    pc: 33           // we'll go by line numbers instead of bytecode
    body: [...]      // lines 1-33
    callStack: {}
    env: {
        feedCo: Lambda
        process: Lambda
        sourceCo: Lambda
        fc: Coroutine
        r1: 1
        r2: 2
        r3: null
    }
}

processCoFun {
    parent: MAIN
    state: TERMINATED
    pc: 16
    body: [...]      // lines 9-15
    callStack: {}
    env: {
        sc: Coroutine
        unused: null
        x1: 1
        x2: 2
        x3: null
    }
}

sourceCoFun {
    parent: processCoFun
    state: TERMINATED
    pc: 25
    body: [...]      // lines 22-24
    callStack: {}
    env: {
        unused: null
    }
}
\end{verbatim}
Output: \texttt{End.}
\end{enumerate}

\end{document}