\section{Registration Logic}
\subsection{Overview}
In order to implement the registration logic, we use the class TPCRegistrationHandler as the NetworkHandler for a SocketServer that gets passed to TPCMasterHandler. This way, when we want to register a slave, we call TPCMasterHandler.registerWithMaster which now has a way to give it to RPCRegistrationHandler to parse and check for success. TPCRegistrationHandler does this by parsing the message in the socket that was created by TPCMasterHandler and creating a TPCSlaveInfo object, an object that has all the relevant information for a master to register a slave. It calls a runnable inner class called RegistrationHandler that does most of the work specified above.
\subsection{Correctness Constraints}
\begin{itemize}
\item If successfully registered, RPCRegistrationHandler.handle will throw no error and include in the socket the correct slave info along with a success message.
\item If unsuccessful, there should be no crashes and the correct error message will propogate to let the TPCMasterHandler know that it was unsuccessful.
\end{itemize}
\subsection{Declarations}
\subsection{Descriptions}
\begin{verbatim}
import nu.xom.*
class TPCRegistrationHandler:
    def handle(Socket slave):
        // put this in a try
        runnableHandler = new RegistrationHandler(slave)
        threadPool.addJob(runnableHandler)

    class RegistrationHandler:
        def run():
            try:
                convert KVMessage(slave) and get the message
                slaveInfo = new TPCSlaveInfo(message from stream)
                master.registerSlave(slaveInfo)
                send KVMessage(slave) with a success message
            except some exceptions:
                put an error message into slave socket

class TPCSlaveInfo:
    def TPCSlaveInfo(String info):
        try:
            slaveID = info.substring(0, indexof('@'))
            hostname = info.substring(indexof('@')+1, indexof(':'))
            port = info.substring(indexof(':')+1, info.length)
        except character not found:
            throw new KVException(ERROR_INVALID_FORMAT)

    def Socket connectHost(int timeout):
        try:
            sock = new Socket(hostname, port)
            sock.setSoTimeout(timeout)
            return sock
        except something
            something

    def void closeHost(Socket sock):
        sock.close()

class TPCMasterHandler:
    def registerWithMaster:
        try:
            masterSock = new Socket(masterHostname, 9090)
            create KVMessage with type REGISTER and message slaveID+'@'+server.hostname+':'+server.port
            send it with sendMessage(masterSock)
            make new KVMessage(masterSock) to get response
            if response is no good, throw invalid format KVException
        except some socket error:
            throw new KVException with appropriate message

\end{verbatim}
\subsection{Testing Plan}
\begin{itemize}
\item Construct a slave server and check that its unique parameters match that of one of the master's slaves after it has registered.
\item Construct a slave server with some fauly parameters and make sure that an error message is properly received when we try to register.
\item Construct and register a few concurrently and make sure it still works as defined.
\end{itemize}
