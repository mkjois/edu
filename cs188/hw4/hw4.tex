%%% Please replace what is within each curly bracket with the correct information below. %%%
%%% The first field is already filled in for you.  %%%
\def\ClassName {CS188} % Your course 
\def\NameLast {Jois}  % Your last name
\def\NameFirst {Manohar}  % Your first name
\def\SID{23808180}  % Your SID
\def\Email{m.k.jois@berkeley.edu} % Your pandagrader email
\def\Collaborators{} % Any collaborators


%%%%% FILL IN YOUR ANSWERS TO EACH QUESTION BY REPLACE WHAT IS IN THE CURLY BRACKETS.
\def\AnswerOneAi{\textbf{Put your answer to 1ai here:} 
%%% Begin 1a answer
The variables are $q_1, \ldots, q_6$ for the six questions, which can each take on a value in $\{1,2,3\}$ for John, Kevin and Anna, respectively.
\begin{align*}
q_1 &\in \{1,2,3\} \\
q_2 &\in \{1,2\} \\
q_3 &\in \{1,2,3\} \\
q_4 &\in \{1\} \\
q_5 &\in \{1,3\} \\
q_6 &\in \{1\}
\end{align*}
%%% End 1a answer
}
\def\AnswerOneAii{\textbf{Put your answer to 1aii here:}
%%% Begin 1b answer
\\
\begin{tabular}{cc}
$q_4 \neq 2$ & $q_4 = q_6$ \\
$q_5 \neq 2$ & $q_1 \neq q_2$ \\
$q_6 \neq 2$ & $q_2 \neq q_3$ \\
$q_2 \neq 3$ & $q_3 \neq q_4$ \\
$q_4 \neq 3$ & $q_4 \neq q_5$ \\
$q_6 \neq 3$ & $q_5 \neq q_6$
\end{tabular}
%%% End 1b answer 
}
\def\AnswerOneB{\textbf{Put your answer(s) to 1b here:}\\
%%% Begin 1c answer
\includegraphics[scale=0.07]{figures/q1b}
%%% End 1c answer 
}

\def\AnswerTwoAi{
%%% Use \cross{1} to cross out a number
\begin{tabular}{ccccccc}
P & \cross{1} & \cross{2} & \cross{3} & 4 & 5 & 6\\
B & \cross{1} & 2 & \cross{3} & 4 & \cross{5} & 6\\
C & 1 & 2 & 3 & 4 & 5 & 6\\
K & \cross{1} & 2 & 3 & 4 & 5 & \cross{6}\\
I & 1 & 2 & 3 & 4 & 5 & 6\\
M & \cross{1} & \cross{2} & \cross{3} & \cross{4} & 5 & 6\\
\end{tabular}
}

\def\AnswerTwoAii{
%%% Replace '\Circle' with '\CIRCLE' to color in the bubble
\Circle \hspace{0.2cm} P \hfill
\Circle \hspace{0.2cm} B \hfill
\Circle \hspace{0.2cm} C \hfill
\Circle \hspace{0.2cm} K \hfill
\Circle \hspace{0.2cm} I \hfill
\CIRCLE \hspace{0.2cm} M \hfill\\
}

\def\AnswerTwoAiii{
%%% Use \cross{1} to cross out a number
\begin{tabular}{ccccccc}
P &   &   &   &  &   &  6\\
B & \cross{1} & \cross{2} & \cross{3} & 4 & \cross{5} & \cross{6}\\
C & 1 & 2 & 3 & 4 & 5 & \cross{6}\\
K & \cross{1} & 2 & 3 & 4 & 5 & \cross{6}\\
I & 1 & 2 & 3 & 4 & 5 & \cross{6}\\
M & \cross{1} & \cross{2} & \cross{3} & \cross{4} & 5 & \cross{6}\\
\end{tabular}
}

\def\AnswerTwoAivPartOne{
\begin{tabular}{|c|c|}
\hline
Variable & \# violated\\
\hline
P & 0%Insert number here% 
\\\hline
B & 0%Insert number here% 
\\\hline
C & 1%Insert number here% 
\\\hline
K & 0%Insert number here% 
\\\hline
I & 2%Insert number here% 
\\\hline
M & 0%Insert number here% 
\\\hline
\end{tabular}
}

\def\AnswerTwoAivPartTwo{
%%% Answer by placing 'x' in correct grids
\begin{tabular}{|c|c|c|c|c|c|c|}
\hline & 1 & 2 & 3 & 4 & 5 & 6 
\\\hline
P & & & & & &
\\\hline
B & & & & & &
\\\hline
C & & x & & & &
\\\hline
K & & & & & &
\\\hline
I & & x & x & x & x &
\\\hline
M & & & & & &
\\\hline
\end{tabular}
}

%%% Replace '\Circle' with '\CIRCLE' to color in the bubble
\def\AnswerThreeA{
\CIRCLE \hspace{0.2cm} Neither MRV nor LCV can have an effect.\\\\ &
\Circle \hspace{0.2cm} Only MRV can have an effect.\\\\ &
\Circle \hspace{0.2cm} Only LCV can have an effect .\\\\ &
\Circle \hspace{0.2cm} Both MRV and LCV can have an effect.\\\\
}
\def\AnswerThreeB{
\Circle \hspace{0.2cm} Neither MRV nor LCV can have an effect.\\\\ &
\CIRCLE \hspace{0.2cm} Only MRV can have an effect.\\\\ &
\Circle \hspace{0.2cm} Only LCV can have an effect .\\\\ &
\Circle \hspace{0.2cm} Both MRV and LCV can have an effect.\\\\
}
\def\AnswerThreeC{
\Circle \hspace{0.2cm} Neither MRV nor LCV can have an effect.\\\\ &
\CIRCLE \hspace{0.2cm} Only MRV can have an effect.\\\\ &
\Circle \hspace{0.2cm} Only LCV can have an effect .\\\\ &
\Circle \hspace{0.2cm} Both MRV and LCV can have an effect.\\\\
}
\def\AnswerThreeD{
\Circle \hspace{0.2cm} Neither MRV nor LCV can have an effect.\\\\ &
\CIRCLE \hspace{0.2cm} Only MRV can have an effect.\\\\ &
\Circle \hspace{0.2cm} Only LCV can have an effect .\\\\ &
\Circle \hspace{0.2cm} Both MRV and LCV can have an effect.\\\\
}
\def\AnswerThreeE{
\Circle \hspace{0.2cm} Neither MRV nor LCV can have an effect.\\\\ &
\CIRCLE \hspace{0.2cm} Only MRV can have an effect.\\\\ &
\Circle \hspace{0.2cm} Only LCV can have an effect .\\\\ &
\Circle \hspace{0.2cm} Both MRV and LCV can have an effect.\\\\
}

%%%%%%%%%%%%%%%%%%%%%% DO NOT CHANGE ANYTHING BELOW THIS LINE %%%%%%%%%%%%%%%%%%%%%%%%%
\documentclass[written]{cs188}

\usepackage{verbatim}
\usepackage{fancyhdr}
\usepackage{booktabs}
\usepackage{setspace}
\usepackage{amsmath,mathrsfs}
\usepackage{multicol}
\usepackage{amssymb}
\usepackage{algpseudocode}
\usepackage{graphicx}
\usepackage{caption}
\usepackage{subcaption}
\usepackage{array}
\usepackage{xcolor}
\usepackage{float}
\usepackage{enumitem}
\usepackage{mathcomp}
\usepackage{tabularx}
\usepackage{wasysym}
\usepackage{pbox}
\usepackage{tikz}
\usetikzlibrary{matrix}
\usepackage[normalem]{ulem}
\usepackage{multirow}

% Bubbles for multiple choice questions
\newcommand{\mcqbubble}{\bigcirc}
\newcommand{\mcqbubblefill}{\Large\newmoon}

% shorthand
\newcommand{\mcqb}{$\bigcirc$\ \ }
\newcommand{\mcqs}{\solution{\mcqb}{$\Large\newmoon$\ \ }}

% pruning
\newcommand{\prune}{\includegraphics[width=0.2in]{figures/red_x}}
\title{Written HW4}

\begin{document}

\newpage
\q{20}{Decision Trees}

You are given the following data sets

\begin{tabular}{c c c c c c}
(i)
&
\begin{tabular}{|c | c | c |}
\hline
$X_1$ & $X_2$ & $Y$\\
\hline
1 & 1 & +\\
\hline
4 & 5 & +\\
\hline
4 & 5 & -\\
\hline
5 & 5 & +\\
\hline
\end{tabular}
&
(ii)
&
\begin{tabular}{|c | c | c |}
\hline
$X_1$ & $X_2$ & $Y$\\
\hline
1 & 1 & +\\
\hline
4 & 3 & +\\
\hline
4 & 5 & -\\
\hline
5 & 5 & +\\
\hline
\end{tabular}
&
(iii)
&
\begin{tabular}{|c | c | c |}
\hline
$X_1$ & $X_2$ & $Y$\\
\hline
1 & 1 & +\\
\hline
4 & 2 & -\\
\hline
4 & 5 & -\\
\hline
5 & 5 & +\\
\hline
\end{tabular}
\end{tabular}


\begin{question}{}
  [3 pts] Which data sets are linearly separable?
\end{question}
\begin{minipage}{\textwidth}
    \solution{\ \vspace{2cm}} {
    Data set (ii)
    }
\end{minipage}

\begin{question}{}
  [3 pts] Which data sets have label noise?
\end{question}
\begin{minipage}{\textwidth}
    \solution{\ \vspace{2cm}} {
    Data set (i)
    }
\end{minipage}

\begin{question}{}
  [3 pts] Which data sets can be fit exactly by a decision tree?
\end{question}
\begin{minipage}{\textwidth}
    \solution{\ \vspace{2cm}} {
    Data sets (ii) and (iii)
    }
\end{minipage}

\begin{question}{}
  [5 pts] A 1-decision-list is a decision tree in which the "yes" branch of every binary test is a leaf node. For a continuous attribute $X_j$, a test can be either $X_j > c$ or $X_j < c$. Continuous attributes can appear in multiple tests. Pick a data set and show a decision list that fits it exactly.
\end{question}
\begin{minipage}{\textwidth}
    \solution{} { 
    Data set (iii) in pseudocode: \begin{verbatim}
         if x1 < 4 : return +
    else if x2 < 5 : return -
    else if x1 < 5 : return -
    else           : return +
    \end{verbatim}
    }
\end{minipage}

\begin{question}{}
  [6 pts] In the absense of label noise, can any two-class data set in two dimensions be fit exactly by a decision list? Briefly explain why, or give a counterexample.
\end{question}
\begin{minipage}{\textwidth}
    \solution{} {
    \paragraph{} The data set $\{((-1,-1),+),((-1,1),-),((1,1),+),((1,-1),-)\}$ is a counterexample. No horizontal or vertical line (which are the only tests we can impose) creates a region of exactly one class.
    }
\end{minipage}



\newpage
\q{30}{Neural Networks}

In this problem, you are given a simple network that uses the simple linear function $g(x) = mx + b$ (where $m$ and $b$ values are fixed) as the activation function (rather than, for example,  a sigmoid function). You will need to write the $L_2$ norm (squared) loss function, the partial derivative of the loss function, and the gradient descent update rule for certain weights.

\textbf{Single Neuron Example}

We have given you the answers for the loss function, partial derivative, and weight update rule for the following single neuron example. This example should help you understand how to structure your answers to the questions about the slightly more complex network on the following page.

\begin{figure}[h]
\centering
\includegraphics[width=0.25\linewidth]{figs/NN_1.png}
\end{figure}
\begin{flalign*}
a_1 &= x_1 &\\
a_2 &= x_2 &\\
a_3 &= m(w_{13}a_1+w_{23}a_2) + b &
\end{flalign*}
\textbf{Loss function}
\begin{flalign*}
Loss(\boldsymbol{w}) &= ||\boldsymbol{y}-\boldsymbol{h_w}(\boldsymbol{x})||_2^2 &\\
&= (y_1 - a_3)^2 &\\
&= (y_1 - m(w_{13}a_1+w_{23}a_2) - b)^2 &\\
&= (y_1 - m(w_{13}x_1+w_{23}x_2) - b)^2 &
\end{flalign*}
\textbf{Loss partial derivative}
\begin{flalign*}
\frac{\partial }{\partial w_{13}}Loss(\boldsymbol{w}) &= -2(y_1 - a_3) \frac{\partial a_3}{\partial w_{13}} &\\
&= -2(y_1 - a_3)ma_1 &\\
&= -2(y_1 - m(w_{13}x_1+w_{23}x_2) - b)mx_1 &
\end{flalign*}
\textbf{Weight update rule}
\begin{flalign*}
w_{13} &\leftarrow w_{13} - \alpha \frac{\partial }{\partial w_{13}} Loss(\boldsymbol{w})&\\
&= w_{13} + \alpha 2(y_1 - m(w_{13}x_1+w_{23}x_2) - b)mx_1 &
\end{flalign*}

(Question continued on next page)

\newpage
\textbf{Simple Neural Network with Linear Activation Function}

Given the following neural network (again, using the simple linear activation function $g(x) = mx + b$):

\begin{figure}[h]
\centering
\includegraphics[width=0.4\linewidth]{figs/NN_2.png}
\end{figure}

\begin{question}{[5 pts]}
Write the loss function, $Loss(\boldsymbol{w})$, in terms of $x_1$, $y_1$, $w_{12}$, $w_{13}$, $w_{24}$, $w_{34}$, $m$, and $b$.

\begin{minipage}{\textwidth}
    \solution{} {\begin{align*}
    Loss(\boldsymbol{w}) &= (y_1-a_4)^2\\
    &= (y_1-m(w_{24}a_2+w_{34}a_3)-b)^2\\
    &= (y_1-m(w_{24}(mw_{12}a_1+b)+w_{34}(mw_{13}a_1+b))-b)^2\\
    &= (y_1-m^2w_{12}w_{24}x_1-mw_{24}b-m^2w_{13}w_{34}x_1-mw_{34}b-b)^2
    \end{align*}
    }
\end{minipage}

\end{question}

\begin{question}{[5 pts]}
Write the derivative of the loss function with respect to $w_{24}$, $\frac{\partial}{\partial w_{24}} Loss(\boldsymbol{w})$, in terms of $x_1$, $y_1$, $w_{12}$, $w_{13}$, $w_{24}$, $w_{34}$, $m$, and $b$.

\begin{minipage}{\textwidth}
    \solution{} {\begin{align*}
    \frac{\partial}{\partial w_{24}}Loss(\boldsymbol{w}) &= -2(y_1-a_4)\frac{\partial a_4}{\partial w_{24}}\\
    &= -2ma_2(y_1-a_4)\\
    &= -2m(mw_{12}x_1+b)(y_1-m^2w_{12}w_{24}x_1-mw_{24}b-m^2w_{13}w_{34}x_1-mw_{34}b-b)
    \end{align*}
    }
\end{minipage}

\end{question}

\begin{question}{[2 pts]}
Write the gradient descent update rule for $w_{24}$ with step size $\alpha$ in terms of $\alpha$ $x_1$, $y_1$, $w_{12}$, $w_{13}$, $w_{24}$, $w_{34}$, $m$, and $b$.

\begin{minipage}{\textwidth}
    \solution{} {\begin{align*}
    w_{24} &\leftarrow w_{24}-\alpha\frac{\partial}{\partial w_{24}} Loss(\boldsymbol{w})\\
    &= w_{24}+2\alpha m(mw_{12}x_1+b)(y_1-m^2w_{12}w_{24}x_1-mw_{24}b-m^2w_{13}w_{34}x_1-mw_{34}b-b)
    \end{align*}
    }
\end{minipage}

\end{question}

\begin{question}{[5 pts]}
Write the derivative of the loss function with respect to $w_{12}$, $\frac{\partial}{\partial w_{12}} Loss(\boldsymbol{w})$, in terms of $x_1$, $y_1$, $w_{12}$, $w_{13}$, $w_{24}$, $w_{34}$, $m$, and $b$.

\begin{minipage}{\textwidth}
    \solution{} {\begin{align*}
    \frac{\partial}{\partial w_{12}}Loss(\boldsymbol{w}) &= -2(y_1-a_4)\frac{\partial a_4}{\partial w_{12}}\\
    &= -2(y_1-a_4)(mw_{24}\frac{\partial a_2}{\partial w_{12}}+mw_{34}\frac{\partial a_3}{\partial w_{12}})\\
    &= -2(y_1-a_4)(mx_1+0)\\
    &= -2mx_1(y_1-m^2w_{12}w_{24}x_1-mw_{24}b-m^2w_{13}w_{34}x_1-mw_{34}b-b)
    \end{align*}
    }
\end{minipage}

\end{question}

\begin{question}{[2 pts]}
Write the gradient descent update rule for $w_{12}$ with step size $\alpha$ in terms of $\alpha$ $x_1$, $y_1$, $w_{12}$, $w_{13}$, $w_{24}$, $w_{34}$, $m$, and $b$.

\begin{minipage}{\textwidth}
    \solution{} {\begin{align*}
    w_{12} &\leftarrow w_{12}-\alpha\frac{\partial}{\partial w_{12}} Loss(\boldsymbol{w})\\
    &= w_{12}+2\alpha mx_1(y_1-m^2w_{12}w_{24}x_1-mw_{24}b-m^2w_{13}w_{34}x_1-mw_{34}b-b)
    \end{align*}
    }
\end{minipage}
\end{question}

\newpage

\begin{question}{[3 pts]}
Because this network has a linear activation function, there is an equivalent network that has no hidden layers. 1) Draw a new (very simple) network that has no hidden layers but computes exactly the same function. 2) Write the new weight explicitly in terms of the $w_{12}$, $w_{13}$, $w_{24}$, $w_{34}$, $m$, and $b$. 3) You will need to adjust the linear activation function, $g_2(x) = m_2x + b_2$. Write the new $m_2$ and $b_2$ values in terms of $w_{12}$, $w_{13}$, $w_{24}$, $w_{34}$, $m$, and $b$.

\begin{minipage}{\textwidth}
    \solution{} {\begin{verbatim}
    x1 (1) a1 -----------> (2) a2', y1
                   w'
    \end{verbatim}
    \begin{align*}
    a2' &= a4\\
    &= m^2w_{12}w_{24}x_1+mw_{24}b+m^2w_{13}w_{34}x_1+mw_{34}b+b\\
    &= m^2(w_{12}w_{24}+w_{13}w_{34})x_1 + (mw_{24}b+mw_{34}b+b)\\
    w' &= w_{12}w_{24}+w_{13}w_{34}\\
    m_2 &= m^2\\
    b_2 &= mw_{24}b+mw_{34}b+b
    \end{align*}
    }
\end{minipage}
\end{question}

\textbf{General Neural Network with Linear Activation Function}

\vspace{3mm}
Consider a new neural network with $n$ input nodes, $n$ output nodes, one hidden layer with $h$ nodes, and the linear activation function $g(x) = mx + b$ at each hidden and output node. The nodes between each layer are fully connected with weights $w_{ij}$ from the $i$-th input node to the $j$-th hidden node and weights $w_{jk}$ from the $j$-th hidden node to the $k$-th output node.

\vspace{3mm}
\begin{question}{[5 pts]}
Because this general network has a linear activation function, there is an equivalent network that has no hidden layers that computes exactly the same function. 1) Write an equation for the weight $w_{ik}$ from the $i$-th input node to the $k$-th output node explicitly in terms of the previous network weights ($w_{ij}$, $w_{jk}$), $m$, and $b$. 2) You will need to adjust the linear activation function, $g_2(x) = m_2x + b_2$. Write the new $m_2$ and $b_2$ values in terms of the previous network weights ($w_{ij}$, $w_{jk}$), $m$, and $b$.

\begin{minipage}{\textwidth}
    \solution{} {\begin{align*}
    y_k &= m\sum_{j=1}^h(w_{jk}(m\sum_{i=1}^n(w_{ij}x_i)+b))+b\\
    &= m^2\left(\sum_{j=1}^h\sum_{i=1}^n w_{ij}w_{jk}x_i\right) + \left(\sum_{j=1}^h\sum_{i=1}^n mw_{jk}b\right) + b\\
    w_{ik} &= \sum_{j=1}^h w_{ij}w_{jk}\\
    m_2 &= m^2\\
    b_{2,k} &= b\left(1+mn\sum_{j=1}^h w_{jk}\right)
    \end{align*}
    }
\end{minipage}
\end{question}

\begin{question}{[3 pts]}
What effect does removing the hidden layer from this general network have on the number of weights? Specifically, in terms of $n$ and $h$, how many weights are there before and after removing the hidden layer? Discuss in particular the case when $h << n$.

\begin{minipage}{\textwidth}
    \solution{} {
    \paragraph{} Before removing the hidden layer there are $hn$ weights on each side of it for a total of $2hn$ weights. After removing it, there are $n^2$ weights, one for each pair of input/output. When $h<<n$ this essentially increases the number of weights by a factor of $\frac n2$.
    }
\end{minipage}

\end{question}





\newpage
\begin{problem}[]{The Value of Games}

Pacman is the model of rationality and seeks to maximize his expected utility,\\
but that doesn't mean he never plays games.


\begin{question}[4] \textbf{A Costly Game.}
  Pacman is now stuck playing a new game with only costs and no payoff. Instead
  of maximizing expected utility $V(s)$, he has to minimize expected costs
  $J(s)$.  In place of a reward function, there is a cost function $C(s,a,s')$
  for transitions from $s$ to $s'$ by action $a$. We denote the discount
  factor by $\gamma \in (0,1)$. $J^*(s)$ is the expected cost incurred by the
  optimal policy. Which one of the following equations is satisfied by $J^*$?

  \ThreeA
\end{question}


\begin{question}[4] \textbf{It's a conspiracy again!}
  The ghosts have rigged the costly game so that once Pacman takes an action
  they can pick the outcome from all states $s' \in S'(s,a)$, the set of all
  $s'$ with non-zero probability according to $T(s,a,s')$. Choose the correct
  Bellman-style equation for Pacman against the adversarial ghosts.

  \ThreeB
\end{question}
\end{problem}

\end{document}