\section{TPCMaster 2PC Logic}
\subsection{Overview}
We will implement the bulk of the 2 Phase Commit logic in the \texttt{handleGet()} and \texttt{handleTPCRequest()} methods in TPCMaster.
\subsection{Correctness Constraints}
\begin{itemize}
\item \texttt{handleGet()} follows the procedure outlined in the specifications (cache $\rightarrow$ primary $\rightarrow$ secondary $\rightarrow$ die)
\item TPCMaster should allow multiple concurrent GET requests.
\item TPCMaster should block requests if there is a PUT/DEL request being handled (with a lock)
\item TPCMaster should not take any requests until all slaves are registered
\item TPCMaster should throw an exception in the case that both slaves fail
\end{itemize}
\subsection{Declarations}
\begin{itemize}
\item readers - tracks the number of GET requests in progress
\item writers - tracks the number of PUT/DEL requests in progress
\end{itemize}
\subsection{Descriptions}
\begin{verbatim}
def handleTPCRequest(KVMessage msg, boolean isPutReq):
    while (numSlaves < slavePool.size()):
        wait()
    while (writers >= 0 || readers >= 0):
        wait()

    writers +=1
    rep1 = findFirstReplica(key)
    rep2 = findSuccessor(first)
    sock1 = rep1.connectHost(TIMEOUT)
    sock2 = rep2.connectHost(TIMEOUT)
    msg.sendMessage(sock1)
    msg.sendMessage(sock2)
    abortmsg = new KVMessage()
    abortmsg.type = KVConstants.ABORT
    commitmsg = new KVMessage()
    commitmsg.type = KVConstants.COMMIT
    abort = false

    try:
        reply1 = new KVMessage(sock1, TIMEOUT)
        reply2 = new KVMessage(sock2, TIMEOUT)
        if (reply1.type is abort or reply2.type is abort):
            abort = true
    catch (KVException e):
        abort = true

    if (abort):
        abortmsg.sendMessage(sock1)
        abortmsg.sendMessage(sock2)
    else:
        commitmsg.sendMessage(sock1)
        commitmsg.sendMessage(sock2)

    // throws exception if we don't receive ACK's
    try:
        ack1 = new KVMessage(sock1, TIMEOUT)
        ack2 = new KVMessage(sock2, TIMEOUT)
    catch(KVException e):
        throw new KVException(KVConstants.ERROR_INVALID_FORMAT);
    
    writers -= 1
    notify()

def handlGet(KVMessage msg)
    while (numSlaves < slavePool.size()):
        wait()
    while (writers != 0):
        wait()

    readers +=1
    val = cache.get(msg.key)
    if (val != null)
         return val
    rep1 = findFirstReplica(msg.key)
    sock1 = rep1.connectHost(TIMEOUT)
    msg.sendMessage(sock1)
    
    try:
        reply1 = new KVMessage(sock1, TIMEOUT)
        if (reply1.type == KVConstants.RESP):
            if (reply1.message == null):
                readers -=1 
                notify()
                return reply1.value 
    catch(KVException e):
    
    rep2 = findSuccessor(first)
    sock2 = rep2.connectHost(TIMEOUT) 
    msg.sendMessage(sock2)
     
    try:      
        reply2 = new KVMessage(sock2, TIMEOUT)
        readers -=1
        notify()
        if (reply2.type == KVMessage.resp):
            if (reply2.message != null):
                throw new KVException(KVConstants.ERROR_NO_SUCH_KEY)                  
            return reply2.value 
    catch(KVException e):
        throw new KVException(KVConstants.ERROR_COULD_NOT_CONNECT)
    
      
    
\end{verbatim}
\subsection{Testing Plan}
We will try all the ways our system can fail and make sure we recover/handle correctly.
\begin{itemize}
\item Try lookups in the midst of a slave failure
\item Try with cache hits and misses
\item Try with slaves that simply don't respond or respond late
\item Try calling without registering all slaves
\end{itemize}
