\documentclass[11pt]{article}
\usepackage{textcomp,geometry,graphicx,verbatim}
\usepackage{fancyhdr}
\usepackage{amsmath,amssymb,enumerate}
\pagestyle{fancy}
\newcounter{problemnumber}
\def\Name{Manohar Jois}
\def\Homework{1} % Homework number - make sure to change for every homework!
\def\Session{Spring 2014}

% Extra commands
\let\origleft\left
\let\origright\right
\renewcommand{\left}{\mathopen{}\mathclose\bgroup\origleft}
\renewcommand{\right}{\aftergroup\egroup\origright}
\newcommand{\N}{\mathbb{N}}
\newcommand{\Z}{\mathbb{Z}}
\newcommand{\R}{\mathbb{R}}
\newcommand{\Q}{\mathbb{Q}}
\newcommand{\C}{\mathbb{C}}
\newcommand{\p}[1]{\left(#1\right)}
\renewcommand{\gcd}[1]{\text{gcd}\p{#1}}
\renewcommand{\deg}[1]{\text{deg}\p{#1}}
\renewcommand{\log}[1]{\text{log}\p{#1}}
\newcommand{\logb}[2]{\text{log}_{#1}\p{#2}}
\newcommand{\BigOh}[1]{O\p{#1}}
\newcommand{\BigOmega}[1]{\Omega\p{#1}}
\newcommand{\BigTheta}[1]{\Theta\p{#1}}

\title{CS170--Spring 2014 --- Solutions to Homework \Homework}
\author{\Name}
\lhead{CS170--\Session\  Homework \Homework\ \Name\ Problem \theproblemnumber}

\begin{document}
\maketitle
\setcounter{problemnumber}{0}

\section*{Problem 1}
\stepcounter{problemnumber}
I understand the course policies.


\newpage
\section*{Problem 2}
\stepcounter{problemnumber}
\begin{enumerate}[(a)]
\item $f(n) \in \BigOmega{g(n)}$: A polynomial with a greater exponent in the leading term will always grow faster after a certain x-value no matter by what scalar the other polynomial is multiplied by.
\item $f(n) \in \BigTheta{g(n)}$: As $n$ grows arbitrarily large, the $\log n$ terms grow extremely insignificant compared to the $n$ terms, and so each function can upper-bound the other.
\item $f(n) \in \BigTheta{g(n)}$: The fastest growing term in each function is $n^3$ in both cases. For example, we can multiply each function by $2$ separately and see that it would upper-bound the other function as $n$ grows arbitrarily large.
\item $f(n) \in \BigTheta{g(n)}$: Because $f(n) = \log{500} + \log{n}$ and $g(n) = \log{8} + \log{n}$, both functions are $\log{n}$ plus a constant and so each one can bound the other.
\item $f(n) \in \BigTheta{g(n)}$: Because $g(n) = 7\log{n}$, so the functions only differ by a scalar multiple and are of the same order.
\item $f(n) \in \BigTheta{g(n)}$: Notice that $f(n) = n(5\log{5}) + 5n\log{n}$, so $f$ is an upper bound on $g$, but also notice that $100 \cdot g(n) = n\log{n^{100}} = 50n\log{n} + 50n\log{n}$, which is an upper bound on $f$.
\item $f(n) \in \BigOh{g(n)}$: It's clear that $n^4(\log{n})^3 > n^3/\log{n}$ as $n$ grows large, and it's also clear that $n^3/\log{n} < n^3 < n^4 < n^4(\log{n})^3$.
\item $f(n) \in \BigOh{g(n)}$: Notice that $g(n) = n(n^{0.01})$, so we only have to compare $(\log{n})^5$ with $n^{0.01}$, and polynomials of base $n$ grow faster in the long run than polynomials of base $\log{n}$, no matter the exponent (as long as it's positive).
\item $f(n) \in \BigOh{g(n)}$: Same reason as (h).
\item $f(n) \in \BigOmega{g(n)}$: Same reason as (h), since $f(n) = n^{1/3}$.
\item $f(n) \in \BigOh{g(n)}$: Notice that $n \log n$ is in $\BigOh{(\log n)^c}$ (for a constant $c$), which is clearly in $\BigOh{(\log n)^{\log n}}$, but the reverses are not true.
\item $f(n) \in \BigOh{g(n)}$: Notice that $g(n) = (2^{\logb 23})^{\logb 2n} = n^{\logb 23}$, which is different from $f(n)$ only in the exponent, which is greater.
\item $f(n) \in \BigOmega{g(n)}$: Look at the inequalities $n4^n < c5^n$ and $5^n < dn4^n$, for constants $c,d$. These are respectively reduced to $n/c < (5/4)^n$ and $(5/4)^n < dn$. As $n$ grows arbitrarily large, the first inequality holds while the second doesn't.
\item $f(n) \in \BigOh{g(n)}$: Notice that $g(n) = (2^3)^n = 8^n$, which is an exponential function with a greater base, and so is of a higher order.
\item $f(n) \in \BigTheta{g(n)}$: Multiply $f(n)$ by any $c < 1/7$ to impose a lower bound on $g$, and multiply $g(n)$ by any $d > 7$ to impose an upper bound on $f$.
\item $f(n) \in \BigOh{g(n)}$: Notice that $g(n) = (7^{\log n})^{\log n}$, so we only have to compare $\log n$ and $7^{\log n}$, and the latter grows much faster, as it is much closer to exponential growth.
\item $f(n) \in \BigOh{g(n)}$: Let's start with the fact that $20! > 8^{20}$ (easily checked, I used wolframalpha.com). We can show that factorial grows faster by pointing out that each additional factor added to the factorial as $n$ increases is greater than $8$. Even if we multiply $8^n$ by some constant $c$, that constant will be in the factorial expression somewhere to even it out.
\end{enumerate}


\newpage
\section*{Problem 3}
\stepcounter{problemnumber}
\begin{enumerate}[1.]
\item \begin{align*}
\sum_{i=1}^n i^k &= 1 + 2^k + \cdots + n^k \\
&\leq n^k + n^k + \cdots + n^k \\
&= n(n^k) \\
&= n^{k+1}
\end{align*}
So it is in $\BigOh{n^{k+1}}$.
\begin{align*}
\sum_{i=1}^n i^k &= 1 + 2^k + \cdots + n^k \\
&\geq (\frac n2)^k + (\frac n2 + 1)^k + \cdots + n^k \\
&\geq \frac n2 (\frac n2)^k \\
&= (\frac n2)^{k+1} \\
&= \frac 1{2^{k+1}} \cdot n^{k+1}
\end{align*}
It is also in $\BigOmega{n^{k+1}}$ and therefore in $\BigTheta{n^{k+1}}$.
\\
\item \begin{align*}
\log{n!} &= \log{1\cdot2\cdot3\cdots n} \\
&\leq \log{n\cdot n\cdot n\cdots n} \\
&= \log{n^n} \\
&= n\log n
\end{align*}
So it is in $\BigOh{n \log n}$.
\begin{align*}
\log{n!} &= \log{1\cdot2\cdot3\cdots n} \\
&\geq \log{\frac n2 \cdot (\frac n2 + 1) \cdots n} \\
&\geq \log{\frac n2 \cdot \frac n2 \cdots \frac n2} \\
&= \frac n2 \log{\frac n2}
\end{align*}
It is also in $\BigOmega{n \log n}$ and therefore in $\BigTheta{n \log n}$.
\newpage
\item
\begin{align*}
\sum_{i=1}^n \frac 1i &= 1 + \frac12 + \frac13 + \cdots + \frac1n \\
&\leq 1+\frac12+\frac12+\frac14+\frac14+\frac14+\frac14+\frac18+\frac18+\frac18+\frac18+\frac18+\frac18+\frac18+\frac18+\cdots
\end{align*}
Let the last expression be $g(n)$. Notice this function is monotonically increasing and at every power of $2$, $g(n)$ is within $1$ of the exponent $k$. Specifically, $g(2^k) = k + \frac1{2^k}$ for $k \geq 0$. If we let $k=\logb2n$, we can substitute to get $g(n) \approx \logb2n + \frac1n \in \BigOh{\log n}$. \\\\
Similarly,
\begin{align*}
\sum_{i=1}^n \frac 1i &= 1 + \frac12 + \frac13 + \cdots + \frac1n \\
&\geq \frac12+\frac14+\frac14+\frac18+\frac18+\frac18+\frac18+\frac1{16}+\frac1{16}+\frac1{16}+\frac1{16}+\frac1{16}+\frac1{16}+\frac1{16}+\frac1{16}+\cdots
\end{align*}
Similar to above, let $f(n)$ be the last expression. Then $f(2^k) = \frac12 k + \frac1{2^{k+1}}$. Again letting $k=\logb2n$, we get $f(n) \approx \frac12(\logb2n + \frac1n) \in \BigOmega{\log n}$. \\\\
Since the original function is upper-bounded by $g(n) \in \BigOh{\log n}$ and lower-bounded by $f(n) \in \BigOmega{\log n}$, it is therefore in $\BigTheta{\log n}$.
\end{enumerate}


\newpage
\section*{Problem 4}
\stepcounter{problemnumber}
\begin{enumerate}[a)]
\item For $c<1$,
\begin{align*}
1 \leq g(n) &= 1 + c + c^2 + \cdots + c^n \\
&= \sum_{i=0}^n c^i \\
&= \frac{1-c^{n+1}}{1-c} \\
&\leq \frac1{1-c}
\end{align*}
$g(n)$ is lower-bounded by the constant $1$ and upper-bounded by the constant $\frac1{1-c}$, so $g(n) \in \BigTheta{1}$ for $c<1$.
\item For $c=1$,
\begin{align*}
g(n) &= 1 + 1 + 1^2 + 1^3 + \cdots + 1^n \\
&= n
\end{align*}
$g(n) = n$ so clearly $g(n) \in \BigTheta{n}$ for $c=1$.
\item For $c>1$,
\begin{align*}
c^n \leq g(n) &= c^n + c^{n-1} + \cdots + c^2 + c + 1 \\
&= \sum_{i=0}^n c^i \\
&= \frac{c^{n+1}-1}{c-1} \\
&= \frac{c^{n+1}}{c-1} - \frac1{c-1} \\
&\leq \frac c{c-1} \cdot c^n
\end{align*}
$g(n)$ is lower-bounded by $c^n$ and upper-bounded by $c^n$ times the constant $\frac c{c-1}$, so $g(n) \in \BigTheta{c^n}$ for $c>1$.
\end{enumerate}


\newpage
\section*{Problem 5}
\stepcounter{problemnumber}
Proof by strong induction on $n$: \\\\
\textbf{Base Case:} For $n=1$, it is true that $\gcd{F_2,F_1} = \gcd{a,a} = a$. \\\\
\textbf{Hypothesis:} Assume that for some $k \geq 1$, $\gcd{F_{n+1},F_n} = a$ for all $n\leq k$. \\\\
\textbf{Induction:} Our hypothesis allows us to let $F_{n-1}=pa$, $F_n=qa$, $F_{n+1}=ra$. From this is is clear that $pa + qa = ra$, so $p+q=r$ and thus $q<r$. Also from our hypothesis $q,r$ must be coprime, otherwise $\gcd{F_{n+1},F_n} > a$, which is a contradiction. \\\\
Now to find $\gcd{F_{n+2},F_{n+1}}$, we use the following lemma, which was proved in the cited text: \\\\
\textbf{Lemma:} If $d$ divides both $a$ and $b$, and $d=ax+by$ for some integers $x$ and $y$, then necessarily $d = \gcd{a,b}$ (Dasgupta, Papadimitriou, Vazirani. \textit{Algorithms} 21). \\\\
Clearly $a$ divides both $F_{n+2} = (q+r)a$ and $F_{n+1} = ra$. \\
Does $a=(q+r)ax+ray$ for some integers $x,y$? \\
Does $1=(q+r)x+ry$ for some integers $x,y$? \\
If $\gcd{q+r,r} = 1$, then yes it does. \\\\
But notice that $\gcd{q+r,r} = \gcd{q+r \bmod r, r} = \gcd{q,r} = 1$, since $q<r$ and the two are already coprime. \\\\
So by the lemma, it is true that $\gcd{F_{n+2},F_{n+1}} = a$.


\newpage
\section*{Problem 6}
\stepcounter{problemnumber}
\begin{enumerate}[a)]
\item The only elements that don't have an inverse modulo $p^n$ are the elements $x$ that satisfy $\gcd{x,p^n} \neq 1$. Since $p$ is prime, $x$ must be a multiple of $p$. There are $p^{n-1}$ multiples of $p$ in the set of $p^n$ elements. This means that $p^n - p^{n-1} = p^{n-1}(p-1)$ elements have an inverse modulo $p^n$.
\item For each subset of size $p^n$ of consecutive elements, there are $p^{n-1}(p-1)$ elements with an inverse. There are $q^m$ such subsets, but one out of every $q$ elements is a multiple of $q$. Thus the number of elements with an inverse modulo $p^nq^m$ is $p^{n-1}(p-1)q^m\frac{q-1}q = p^{n-1}q^{m-1}(p-1)(q-1)$.
\item Since $1$ is always its own inverse modulo $n$ for $n\geq 2$, we are looking for numbers with only one other inverse. Since inverses usually come in pairs, the only way this is possible is if the other inverse is its own inverse. \\\\
Notice that for $n\geq 3$, $\gcd{n,n-1}=1$, since any prime $p$ that divides both $n$ and $n-1$ must also divide their difference $1$ which is not possible. Since numbers coprime to $n$ have an inverse modulo $n$, it follows that for all $n\geq 3$, both $1$ and $n-1$ are distinct and are their own modular inverses. \\\\
So the numbers we need must then satisfy $\gcd{n,k}\neq 1$ for all $k \in \{2,3,\ldots,n-2\}$ for no other number to have an inverse modulo $n$. The only numbers which satisfy this property are $n \in \{3,4,6\}$.
\end{enumerate}

\end{document}
