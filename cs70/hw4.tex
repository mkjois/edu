\documentclass[11pt,fleqn]{article}
\usepackage{cs70,latexsym,epsf}
\usepackage{rotating}
\usepackage{amsmath,amssymb,polynom}
\usepackage[ruled,vlined]{algorithm2e}
\lecture{4}
\def\title{HW \the\lecturenumber, Manohar Jois}
\begin{document}
\maketitle
\section*{Due Monday September 30 at 5 pm}

% !TEX TS-program = pdflatex
% !TEX encoding = UTF-8 Unicode

% This is a simple template for a LaTeX document using the "article" class.
% See "book", "report", "letter" for other types of document.

%\documentclass[11pt]{article} % use larger type; default would be 10pt

%\usepackage[utf8]{inputenc} % set input encoding (not needed with XeLaTeX)

%%% Examples of Article customizations
% These packages are optional, depending whether you want the features they provide.
% See the LaTeX Companion or other references for full information.

%%% PAGE DIMENSIONS
%\usepackage{geometry} % to change the page dimensions
%\geometry{a4paper} % or letterpaper (US) or a5paper or....
% \geometry{margin=2in} % for example, change the margins to 2 inches all round
% \geometry{landscape} % set up the page for landscape
%   read geometry.pdf for detailed page layout information

%\usepackage{graphicx} % support the \includegraphics command and options

% \usepackage[parfill]{parskip} % Activate to begin paragraphs with an empty line rather than an indent

%%% PACKAGES
%\usepackage{amsmath, amsfonts}
%\usepackage{booktabs} % for much better looking tables
%\usepackage{array} % for better arrays (eg matrices) in maths
%\usepackage{paralist} % very flexible & customisable lists (eg. enumerate/itemize, etc.)
%\usepackage{verbatim} % adds environment for commenting out blocks of text & for better verbatim
%\usepackage{subfig} % make it possible to include more than one captioned figure/table in a single float
% These packages are all incorporated in the memoir class to one degree or another...

%%% HEADERS & FOOTERS
%\usepackage{fancyhdr} % This should be set AFTER setting up the page geometry
%\pagestyle{fancy} % options: empty , plain , fancy
%\renewcommand{\headrulewidth}{0pt} % customise the layout...
%\lhead{}\chead{}\rhead{}
%\lfoot{}\cfoot{\thepage}\rfoot{}

\iffalse
%%% SECTION TITLE APPEARANCE
\usepackage{sectsty}
\allsectionsfont{\sffamily\mdseries\upshape} % (See the fntguide.pdf for font help)
% (This matches ConTeXt defaults)

%%% ToC (table of contents) APPEARANCE
\usepackage[nottoc,notlof,notlot]{tocbibind} % Put the bibliography in the ToC
\usepackage[titles,subfigure]{tocloft} % Alter the style of the Table of Contents
\renewcommand{\cftsecfont}{\rmfamily\mdseries\upshape}
\renewcommand{\cftsecpagefont}{\rmfamily\mdseries\upshape} % No bold!


\newcommand{\N}{\mathbb{N}}
\newcommand{\Z}{\mathbb{Z}}
\newcommand{\R}{\mathbb{R}}
\newcommand{\Q}{\mathbb{Q}}
%%% END Article customizations
\fi
\newcommand{\p}[1]{\left(#1\right)}
\renewcommand{\gcd}[1]{\text{gcd}\p{#1}}
\renewcommand{\deg}[1]{\text{deg}\p{#1}}

%%% The "real" document content comes below...

%\title{Homework \#1}
%\date{} % Activate to display a given date or no date (if empty),
         % otherwise the current date is printed 

%\begin{document}
%\maketitle 


\begin{enumerate}

\item  \textbf{Polynomial GCD:}
Let $A(x)$ and $B(x)$ be polynomials (with coefficients in $\R$ or $GF(m)$). 
We say that  $\gcd{A(x), B(x)} = D(x)$ if $D(x)$ divides $A(x)$ and $B(x)$,
and if every polynomial $C(x)$ that divides both $A(x)$ and $B(x)$ also 
divides $D(x)$. 
For example, $\gcd{(x-1)(x+1), (x-1)(x+2)} = x-1$.
Incidentally, $\gcd{A(x), B(x)}$ is the highest degree polynomial
that divides both $A(x)$ and $B(x)$
(and it is only defined up to multiplication by a constant,
i.e. we could also say $\gcd{(x-1)(x+1), (x-1)(x+2)} = 2x-2$).

Consider the following recursive definition $F$:
\begin{itemize}
\item $F(A(x), 0) = A(x)$.
\item If $A(x) = B(x) Q(x) + R(x)$ with $\deg{R(x)} < \deg{B(x)}$, then
$$F\p{A(x), B(x)} = F(B(x), R(x)).$$ 
($\deg{P(x)}$ denotes the degree of $P(x)$.) 
\end{itemize}

\begin{enumerate}
\item Write a recursive program to compute $F(A(x),B(x))$. You may assume you already have a subroutine for dividing two polynomials.
\begin{tabbing}
function $F(A, B)$: \\
\hspace{1cm} if $B=0$: return $A$ \\
\hspace{1cm} $Q, R = divide(A, B)$ \\
\hspace{1cm} return $F(B, R)$
\end{tabbing}
\item Prove that $F(A(x), B(x)) = \gcd{A(x), B(x)}$. \\\\
First we impose the convention that $\deg{A(x)} \geq \deg{B(x)}$ when using $F(A(x), B(x))$.\\
If we let $\gcd{A(x), B(x)} = G(x)$, then we can write $A(x) = C(x)G(x)$ and $B(x) = D(x)G(x)$ for some polynomials $C(x),D(x)$. \\
Then $R(x) = A(x) - B(x)Q(x) = G(x) \cdot (C(x) - D(x)Q(x))$. So clearly the $gcd$ of $A(x), B(x)$ also divides their remainder $R(x)$. \\
By a symmetrical argument, the $gcd$ of $B(x), R(x)$ also divides the higher order polynomial $A(x)$. \\
We have shown that $\gcd{A(x), B(x)} = \gcd{B(x), R(x)}$. It's also trivial that if $B(x) = 0$, then $\gcd{A(x), B(x)} = A(x)$. Putting these two facts together to form a recursive algorithm to compute $\gcd{A(x), B(x)}$ is exactly the same as the recursive definition of $F(A(x), B(x))$, provided that $A(x) = B(x)Q(x) + R(x)$. Therefore $F(A(x), B(x)) = \gcd{A(x), B(x)}$.
\end{enumerate}

\newpage
\item  \textbf{Extended Euclid for polynomials:}
Let $P(x) = x^4 - 1$ and $Q(x) = x^3 + x^2$.
\begin{enumerate}
\item Find polynomials $A(x)$ and $B(x)$ such that $A(x) P(x) + B(x) Q(x) = x + 1$ for all $x$. \\\\
\polylonggcd{x^4 - 1}{x^3 + x^2}
\begin{eqnarray*}
x+1 &=& (x^3 + x^2) - (x^2 - 1)(x+1) \\
    &=& (x^3 + x^2) - ((x^4 - 1)-(x^3 + x^2)(x-1))(x+1) \\
    &=& (x^4 - 1)(-x-1) + (x^3 + x^2)(x^2)
\end{eqnarray*}
$A(x) = {\bf -x-1} \\ B(x) = {\bf x^2}$
\item Prove there are no polynomials $A(x)$ and $B(x)$ such that $A(x) P(x) + B(x) Q(x) = 1$ for all $x$. \\\\
We know from applying Euclid's algorithm in part (a) that $gcd(P(x),Q(x)) = x+1$, so we can say that $P(x) = (x+1)C(x)$ and $Q(x) = (x+1)D(x)$ for some polynomials $C(x)$ and $D(x)$. Then we can deduce the following:
\begin{eqnarray*}
A(x) P(x) + B(x) Q(x) &=& 1 \\
(x+1)A(x)C(x) + (x+1)B(x)D(x) &=& 1 \\
A(x)C(x) + B(x)D(x) &=& \frac 1{x+1} \qquad \text{(1)}
\end{eqnarray*}
We know by the properties of polynomials that addition and multiplication of polynomials results in another polynomial. Since $C(x)$ and $D(x)$ are polynomials, then we can't have both $A(x)$ and $B(x)$ both be polynomials and still satisfy (1).
\end{enumerate}

\newpage
\item \textbf{Sums of polynomials:}
Remember the first statement we showed by induction was $\sum_{k = 1}^n k = n(n+1)/2$. 
Here is an interesting fact: if instead of summing $k$, we sum a polynomial $P(k)$ of degree $d$,
$$\sum_{k = 1}^n P(k) = Q(n),$$ then the sum $Q(n)$ is a polynomial of degree $d+1$. 
Notice that the statement above is a special case, since $P(k) = k$ is a polynomial of degree $1$ 
and $n(n+1)/2$ is a polynomial of degree $2$. 
You may use this general fact in the problems below.
\begin{enumerate}
\item (\textbf{Optional:} attempt this part only to understand the rest of the question.)
If $P(k) = k$, find $Q(1), Q(2), Q(3)$ and then use Lagrange interpolation to find $Q(n)$.
\item Let $P(k) = 3k^2 + 3k + 1$. By the above fact, $Q(n)$ is a polynomial of degree $3$. 
Compute $Q(1), Q(2), Q(3), Q(4)$,
then guess the general form of $Q(n)$ (it is a very simple cubic).
Prove that your guess is correct by using properties of polynomials. \\
\begin{eqnarray*}
Q(1) &=& 3(1)^2 + 3(1) + 1 = {\bf 7} \\
Q(2) &=& Q(1) + 3(2)^2 + 3(2) + 1 = 7 + 19 = {\bf 26} \\
Q(3) &=& Q(2) + 3(3)^2 + 3(3) + 1 = 26 + 37 = {\bf 63} \\
Q(4) &=& Q(3) + 3(4)^2 + 3(4) + 1 = 63 + 61 = {\bf 124}
\end{eqnarray*}
{\bf Claim:} $Q(n) = (n+1)^3 - 1$ \\
Given $\deg{P(k)} = 2$, we can assume by the above general fact that $\deg{Q(n)} = 3$. By Lagrange Interpolation, we know that once we have the 4 values of $Q(1), Q(2), Q(3), Q(4)$, then a degree 3 polynomial that satisfies all 4 $(n,Q(n))$ pairs is unique. Since $Q(n) = (n+1)^3 - 1$ is a degree 3 polynomial that satisfies $Q(1) = 7$, $Q(2) = 26$, $Q(3) = 63$ and $Q(4) = 124$ (easily checked), it is the unique polynomial which satisfies $\displaystyle\sum\limits_{k = 1}^n P(k) = Q(n),$ for $P(k) = 3k^2 + 3k + 1$.
\item Verify your guess in part (b) using induction,
\emph{without} assuming that $Q$ is a polynomial.
(Parts (b) and (c) give two different proofs of the same statement.) \\\\
{\bf Base Case:} $Q(1) = 3(1)^2 + 3(1) + 1 = 7 = (1+1)^3 - 1$ \\
{\bf Induction Hypothesis:} Assume $Q(n) = (n+1)^3 - 1$ \\
{\bf Induction Step:} We are looking for $Q(n+1) = ((n+1)+1)^3 - 1 = (n+2)^3 - 1$
\begin{eqnarray*}
Q(n+1) &=& Q(n) + 3(n+1)^2 + 3(n+1) + 1 \\
       &=& (n+1)^3 - 1 + 3(n+1)^2 + 3(n+1) + 1 \\
       &=& n^3 + 3n^2 + 3n + 1 + 3n^2 + 6n + 3 + 3n + 3 + 1 - 1 \\
       &=& n^3 + 6n^2 + 12n + 8 - 1 \\
       &=& (n+2)^3 - 1
\end{eqnarray*}
\end{enumerate}

\newpage
\item \textbf{Review with WOP}: Suppose that I have $k$
envelopes, numbered $0, 1, \ldots, k-1$, such that envelope $i$ contains $2^i$ dollars.
Using the well-ordering principle, complete the following proof sketch into a proof.

\textbf{Claim}: for any integer $0 \leq n < 2^k$, there 
is a set of envelopes that contain exactly $n$ dollars between them.

\textbf{Proof sketch:} If $n \geq 2^{k-1}$, we include the envelope with $2^{k-1}$ dollars,
otherwise we don't.
We now need to make either $n$ or $n-2^{k-1}$ dollars
using the remaining envelopes. Either way we can do this by induction.

\textbf{Base Case:} Having $k=0$ means that I have no envelopes at all, and the only $n$ which satisfies $0 \leq n < 2^0$ is $n=0$, which I can make from the empty set of the zero envelopes that I have (so the claim is true). \\ We can also check $k=1$, in which case we have one envelope numbered 0 containing $2^0 = 1$ dollar. The $n$ which satisfy $0 \leq n < 2^1$ are $n=0,n=1$. We can make 0 dollars using the empty set of envelopes and 1 dollar using our only envelope.

\textbf{Hypothesis:} Assume that the claim doesn't hold for all $k$, so that there exists a value $j>1$ such that $k=j$ is the smallest value of $k$ where the claim doesn't hold. This means there exists a value $m < 2^j$ that can't be made using the set of envelopes containing $2^0, 2^1, \ldots 2^{j-1}$ dollars.

\textbf{Step:} If $m < 2^{j-1}$ and we couldn't make $m$ using the elements of the set $\{2^0,2^1,\ldots 2^{j-1}\}$, then it implies we couldn't make $m$ using the elements of the subset of that set $\{2^0,2^1,\ldots 2^{j-2}\}$. This means that $k=j-1$ is the smallest value of $k$ where the claim doesn't hold, contradicting our hypothesis. \\ If $2^{j-1} \leq m < 2^j$ and we couldn't make $m$ using elements of the set $\{2^0,2^1,\ldots 2^{j-1}\}$, then it implies we couldn't make $m-2^{j-1} < 2^j - 2^{j-1} = 2^{j-1}$ using the elements of the set $\{2^0,2^1,\ldots 2^{j-2}\}$. This also means that $k=j-1$ is the smallest value of $k$ where the claim doesn't hold. \\ In either case we get a contradiction of our hypothesis that the claim was false, therefore the claim is true.

\end{enumerate}

\end{document}
