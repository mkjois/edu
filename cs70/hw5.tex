\documentclass[11pt,fleqn]{article}
\usepackage{cs70,latexsym,epsf}
\usepackage{rotating}
\usepackage{amsmath,amssymb}
\usepackage[ruled,vlined]{algorithm2e}
\lecture{5}
\def\title{HW \the\lecturenumber, Manohar Jois}
\begin{document}
\maketitle
\section*{Due Monday October 7 at 5 pm}

% !TEX TS-program = pdflatex
% !TEX encoding = UTF-8 Unicode

% This is a simple template for a LaTeX document using the "article" class.
% See "book", "report", "letter" for other types of document.

%\documentclass[11pt]{article} % use larger type; default would be 10pt

%\usepackage[utf8]{inputenc} % set input encoding (not needed with XeLaTeX)

%%% Examples of Article customizations
% These packages are optional, depending whether you want the features they provide.
% See the LaTeX Companion or other references for full information.

%%% PAGE DIMENSIONS
%\usepackage{geometry} % to change the page dimensions
%\geometry{a4paper} % or letterpaper (US) or a5paper or....
% \geometry{margin=2in} % for example, change the margins to 2 inches all round
% \geometry{landscape} % set up the page for landscape
%   read geometry.pdf for detailed page layout information

%\usepackage{graphicx} % support the \includegraphics command and options

% \usepackage[parfill]{parskip} % Activate to begin paragraphs with an empty line rather than an indent

%%% PACKAGES
%\usepackage{amsmath, amsfonts}
%\usepackage{booktabs} % for much better looking tables
%\usepackage{array} % for better arrays (eg matrices) in maths
%\usepackage{paralist} % very flexible & customisable lists (eg. enumerate/itemize, etc.)
%\usepackage{verbatim} % adds environment for commenting out blocks of text & for better verbatim
%\usepackage{subfig} % make it possible to include more than one captioned figure/table in a single float
% These packages are all incorporated in the memoir class to one degree or another...

%%% HEADERS & FOOTERS
%\usepackage{fancyhdr} % This should be set AFTER setting up the page geometry
%\pagestyle{fancy} % options: empty , plain , fancy
%\renewcommand{\headrulewidth}{0pt} % customise the layout...
%\lhead{}\chead{}\rhead{}
%\lfoot{}\cfoot{\thepage}\rfoot{}

\iffalse
%%% SECTION TITLE APPEARANCE
\usepackage{sectsty}
\allsectionsfont{\sffamily\mdseries\upshape} % (See the fntguide.pdf for font help)
% (This matches ConTeXt defaults)

%%% ToC (table of contents) APPEARANCE
\usepackage[nottoc,notlof,notlot]{tocbibind} % Put the bibliography in the ToC
\usepackage[titles,subfigure]{tocloft} % Alter the style of the Table of Contents
\renewcommand{\cftsecfont}{\rmfamily\mdseries\upshape}
\renewcommand{\cftsecpagefont}{\rmfamily\mdseries\upshape} % No bold!


\newcommand{\N}{\mathbb{N}}
\newcommand{\Z}{\mathbb{Z}}
\newcommand{\R}{\mathbb{R}}
\newcommand{\Q}{\mathbb{Q}}
%%% END Article customizations
\fi
\newcommand{\p}[1]{\left(#1\right)}
\renewcommand{\gcd}[1]{\text{gcd}\p{#1}}
\renewcommand{\deg}[1]{\text{deg}\p{#1}}

%%% The "real" document content comes below...

%\title{Homework \#1}
%\date{} % Activate to display a given date or no date (if empty),
         % otherwise the current date is printed 

%\begin{document}
%\maketitle 

\begin{enumerate}
\item \textbf{Hierarchical secret sharing:}
Suppose that Google's most precious asset
was the simple algorithm at the core of Google search\footnote{Disclaimer: 
there is no simple algorithm at the core of Google search.}.
This secret is closely guarded,
and indeed no single individual at Google knows the algorithm.
In fact, no individual at Google knows anything about the algorithm,
they just oversee the machines that run it.

In the case of an emergency, Google would like to be able to recover this core algorithm.
Google is run by its president and board of $9$ directors,
and they would like a group including
the president and a majority of the members of the board to be able to recover the algorithm.
But they would also like to ensure that no other group can learn \emph{anything} about the algorithm.

Describe a scheme which has the desired property.
You should specify some data to give to the president
and to each member of the board,
such that the president and $5$ members of the board can recover
the secret algorithm but such that any group which doesn't contain the president,
or any group which has only 4 members of the board,
can't learn anything about the algorithm.

\textbf{Solution:} Let $P(x)$ be a degree-4 polynomial $\bmod N$, where $N$ is a large prime, such that $P(0)$ is the secret algorithm.
Now add a constant offset $c$ to get a polynomial $Q(x) = P(x)+c$.
Give the offset $c$ to the president and $Q(1),Q(2),\ldots,Q(9)$ to the nine board members.
Any group of 5 or more board members can recover the polynomial $Q(x)$ but they need the president to give the offset $c$ to recover $P(x)$.
The president has $c$ but has no way of recovering $Q(x)$ without a majority of his board.

\newpage
\item \textbf{Counting:}
Consider the set of degree $\leq d$ polynomials $\bmod p$, for some prime $p$.
Fix $k < d$, and let $0 < a_1 < a_2 < \ldots < a_k < p$ be integers.
\begin{enumerate} 
\item How many degree $\leq d$ polynomials $P$ are there such that $P(a_1) = P(a_2) = \cdots = P(a_k) = 0$? \\
$P$ is of the form $P(x) = (x-a_1)(x-a_2)\cdots(x-a_k)(b_{d-k}x^{d-k} + b_{d-k-1}x^{d-k-1} + \cdots + b_0x^0)$, where $b_{d-k},\ldots,b_0$ are arbirtary constants $\bmod p$. Any constant $c$ multiplied by $P$ will factor into the arbitrary $b_i$, and this polynomial is guaranteed to be of degree $\leq d$ with roots at the $a_i$. This leaves $d-k+1$ arbitrary constants $b_i$ each with $p$ possible values, so there are $p^{d-k+1}$ possible polynomials.
\item How many polynomials are there of the form $P(x) = x^d + Q(x)$, where $Q$ has degree $\leq d-1$,
such that $P(a_1) = P(a_2) = \cdots = P(a_k) = 0$? \\
Again we write $P(x)$ as in part (a) but this time we know $b_{d-k} = 1$. The only way to get an $x^d$ term in $P(x)$ is to multiply $x^{d-k}$ by all the $x$'s in the $(x-a_i)$ terms, so the coefficient of $x^{d-k}$ is 1 and the rest of the polynomial produces a $Q(x)$ of degree $\leq d-1$. There are now $d-k$ arbitrary constants $b_i$ each with $p$ possible values, so we have $p^{d-k}$ possible polynomials.
\item How many degree $\leq d$ polynomials $P$ are there such that $P(a_1) \neq 0, P(a_2) \neq 0, \ldots, P(a_k) \neq 0$? \\
A unique polynomial of degree $\leq d$ can be determined by a set of $d+1$ points each with $p$ possible values. But for the condition to hold, $k$ of the points can't be valued $0 \bmod p$, leaving each of those points with $p-1$ possible values. Thus there are $(p-1)^k \cdot p^{d-k+1}$ possible polynomials.
\end{enumerate}
(Note: in general, when we talk about a degree $d$ polynomial,
we typically mean to include the degenerate case of a degree $d-1$ polynomial.
This problem specifies $\leq d$ only for clarity.)

\newpage
\item \textbf{Adversarial Secret Sharing:} Suppose 
that I want to break up a secret into shares such that any set of $k$ people can recover the secret,
but I'm also worried that some people might be dishonest and may lie about the secrets they have 
(in order to prevent the other people from recovering the correct secret). 
Show that in the ordinary secret sharing scheme based on polynomials,
any group of $k+2$ people can recover the secret even if one of them is dishonest.

\textbf{Solution:} Let $P(x)$ be the polynomial that determines the secret and each of the shares, and $E(x) = (x-e)$ where $e$ is the person who is dishonest about his share $P(e)$. Let $R(x)$ be the reported value of each share where we don't know for which $x=e$ such that $P(e) \neq R(e)$, and $Q(x) = P(x)E(x)$. \\\\
Observe that $Q(x) = P(x)E(x) = R(x)E(x)$ for all $x \in \{1,2,\ldots,k,k+1,k+2\}$ since for $x=e$, $E(x) = 0$ and for $x \neq e$, $P(x) = R(x)$.
Also observe that if we want any set of $k$ people to recover the secret, $P(x)$ must be a polynomial of degree $k-1$ and thus have $k$ unknown coefficients. So $Q(x)$ is a degree $k$ polynomial with $k+1$ unknown coefficients. \\\\
If we write down the equations $Q(i) = R(i)E(i)$ for all $i \in \{1,2,\ldots,k,k+1,k+2\}$, we will have a system of $k+2$ linear equations of $k+2$ unknowns, which are the coefficients of $Q(x)$ plus the unknown $e$.
Since we constructed $Q(x)$ to be $P(x)(x-e)$, we know it has real coefficients because we can assume $P(x)$ has real coefficients in ordinary secret sharing, and $e$ must be one of $\{1,2,\ldots,k,k+1,k+2\}$. So our system of linear equations has a solution. \\\\
Let $Q'(x)$ and $E'(x)$ be the polynomials we reconstructed from our linear equations. Then we observe that both $Q(x) = R(x)E(x)$ and $Q'(x) = R(x)E'(x)$.
Now we consider the polynomials $Q(x)E'(x)$ and $Q'(x)E(x)$. If $E(x) = 0$, then $Q(x) = 0$ and both these polynomials are zero. If $E'(x) = 0$, then $Q'(x) = 0$ and both these polynomials are again zero. Otherwise $Q(x)E'(x) = R(x)E(x)E'(x) = Q'(x)E(x)$ and we can rearrange to get $\frac{Q'(x)}{E'(x)} = \frac{Q(x)}{E(x)} = P(x)$. So the $P(x)$ we get after polynomial division is unique and can be recovered by $k+2$ people even if one is dishonest.

\newpage
\item \textbf{Exam reprise:} For each even numbered question on the midterm,
write up the most convincing 1 to 2 sentence answer to that question that you can.
The midterm will be posted after Thursday.
You are allowed to look at the solutions to the midterm;
we want you to spend some time thinking about the solutions
and trying to synthesize the core ideas.

\textbf{2:} The algorithm originally produces pairings without any rogue couples, and the only person whose preferences change is man 1, who can move woman A a maximum of $n-1$ places down his list (from first to last). Now if there are any rogue couples, man 1 must be a part of each one since only he changed his preferences, and he can only prefer $n-1$ women at most, who may all prefer 1 to their partner, to woman A.

\textbf{4:} Since $\Delta_i(i) = 1 \text{ and } \Delta_i(j) = 0 \text{ for } i \neq j$, $P(x) = \sum_{i=1}^{11} \Delta_i(x) = 1 \text{ for } 1 \leq x \leq 11$.
There must be a unique polynomial of degree-10 or less that goes through these 11 points; but we notice that $P(x) = 1 \text{ for all } x$ fits the criteria and passes through the 11 points so it is the unique polynomial, and therefore $P(20) = 1$.

\end{enumerate}

\end{document}
