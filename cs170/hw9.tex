\documentclass[11pt]{article}
\usepackage{textcomp,geometry,graphicx,verbatim}
\usepackage{fancyhdr}
\usepackage{amsmath,amssymb,enumerate}
\pagestyle{fancy}
\def\Name{Manohar Jois}
\def\Homework{9} % Homework number - make sure to change for every homework!
\def\Session{Spring 2014}

% Extra commands
\let\origleft\left
\let\origright\right
\renewcommand{\left}{\mathopen{}\mathclose\bgroup\origleft}
\renewcommand{\right}{\aftergroup\egroup\origright}
\newcommand{\N}{\mathbb{N}}
\newcommand{\Z}{\mathbb{Z}}
\newcommand{\R}{\mathbb{R}}
\newcommand{\Q}{\mathbb{Q}}
\newcommand{\C}{\mathbb{C}}
\newcommand{\p}[1]{\left(#1\right)}
\renewcommand{\gcd}[1]{\text{gcd}\p{#1}}
\renewcommand{\deg}[1]{\text{deg}\p{#1}}
\renewcommand{\log}[1]{\text{log}\p{#1}}
\renewcommand{\ln}[1]{\text{ln}\p{#1}}
\newcommand{\logb}[2]{\text{log}_{#1}\p{#2}}
\newcommand{\BigOh}[1]{O\p{#1}}
\newcommand{\BigOmega}[1]{\Omega\p{#1}}
\newcommand{\BigTheta}[1]{\Theta\p{#1}}

\title{CS170--Spring 2014 --- Solutions to Homework \Homework}
\author{\Name}
\lhead{CS170--\Session\  Homework \Homework\ \Name\ Problem \theproblemnumber}

\begin{document}
\maketitle
\newcounter{problemnumber}
\setcounter{problemnumber}{0}

\section*{Problem 1}
\stepcounter{problemnumber}
Given 40 feet of fencing, what is the maximum rectangular area you can enclose? The linear program is as follows, where $x_1$ and $x_2$ are the lengths of the sides:
\begin{align*}
\max{x_1x_2} \\
2x_1+2x_2 &\leq 40 \\
x_1 &\geq 0 \\
x_2 &\geq 0
\end{align*}
The feasible region is a triangle with vertices $(0,0)$, $(20,0)$ and $(0,20)$. However we know the optimal solution is $(10,10)$ which produces a square, well known to be the solution to this problem.


\newpage
\section*{Problem 2}
\stepcounter{problemnumber}
% PUT PROBLEM 2 SOLUTION HERE


\newpage
\section*{Problem 3}
\stepcounter{problemnumber}
\begin{enumerate}[(a)]
\item Let $r_1,r_2,r_3$ be g sugar, mg caffeine, and oz water respectively. Let $p_1,p_2$ be cases of Zap Energy and Sugar Water respectively. $$\max{500p_1+200p_2-10r_1-20r_2-r_3}$$
\begin{align*}
10r_1+20r_2+r_3 &\leq b \\
r_1-5p_1-12p_2 &\geq 0 \\
r_2-1p_1-0p_2 &\geq 0 \\
r_3-8p_1-16p_2 &\geq 0 \\
p_1 &\leq 50 \\
p_2 &\leq 80 \\
p_1,p_2,r_1,r_2,r_3 &\geq 0
\end{align*}
\item Let $y_i$ for $1\leq i\leq 6$ be the multipliers of the first six inequalities in the primal LP. After flipping inequalities 2, 3, 4, and multiplication and addition we get: $$(5y_2+y_3+8y_4+y_5)p_1+(12y_2+16y_4+y_6)p_2+(10y_1-y_2)r_1+(20y_1-y_3)r_2+(y_1-y_4)r_3 \leq by_1+50y_5+80y_6$$ The coefficients of the $r_i,p_i$ in this inequality are lower bounded by the values in the primal objective function. All the $y_i$ must be non-negative or else they wouldn't multiply the inequalities correctly. We also want to minimize the right side of the inequality to get the tightest bound on the primal objective. So the dual LP is: $$\min{by_1+50y_5+80y_6}$$
\begin{align*}
5y_2+y_3+8y_4+y_5 &\geq 500 \\
12y_2+16y_4+y_6 &\geq 200 \\
10y_1-y_2 &\geq -10 \\
20y_1-y_3 &\geq -20 \\
y_1-y_4 &\geq -1 \\
y_1,y_2,y_3,y_4,y_5,y_6 &\geq 0
\end{align*}
\item 
\end{enumerate}


\newpage
\section*{Problem 4}
\stepcounter{problemnumber}
Let $x_i$ correspond to Joey's strategies and $y_i$ to Tony's. \\
Tony needs to minimize his expected loss: $\min\{2x_1-x_2,-2x_2,-3x_1+x_2\}=z$. \\
Thus Joey needs to choose $x_i$ to maximize $z$ subject to the following:
\begin{align*}
-2x_1+x_2+z &\leq 0 \\
2x_2+z &\leq 0 \\
3x_1-x_2+z &\leq 0 \\
x_1+x_2 &= 1 \\
x_1,x_2 &\geq 0
\end{align*}
Using the equality constraint we can substitute $x_2=1-x_1$ and get:
\begin{align*}
-3x_1+z &\leq -1 \\
-2x_1+z &\leq -2 \\
4x_1+z &\leq 1 \\
x_1 &\geq 0 \\
x_1 &\leq 1
\end{align*}
Simply plotting this on a 2D graph shows $z$ is maximized when $-4x_1+1=2x_1-2$, which yields $(x_1,x_2)=(1/2,1/2)$ and $z=2(1/2)-2=-1$. \\\\
Dual LP: Joey needs to maximize his expected gain: $\max\{2y_1-3y_3,-y_1-2y_2+y_3\}=w$. \\
Thus Tony needs to choose $y_i$ to minimize $w$ subject to the following:
\begin{align*}
-2y_1+3y_3+w &\geq 0 \\
y_1+2y_2-y_3+w &\geq 0 \\
y_1+y_2+y_3 &= 1 \\
y_1,y_2,y_3 &\geq 0
\end{align*}
To see that this is the dual of the Joey's LP, take the $x_i$ to be the multipliers of the first two inequalities and add them to get $-0.5y_1+y_2+y_3+w \geq 0$. Since $y_1$ must be non-negative, it's clear that it must be zero to minimize $w$, otherwise $w$ would have to be higher to reach the constraint. Therefore $y_2+y_3=1$ and $w \geq -1$. This says Tony's minimum loss is the same as Joey's maximum gain, which is $-1$ dozen pizzas, which is the value of the game. \\\\
We already have Joey's optimal strategy of $(1/2,1/2)$, and to get Tony's we plug in the values of $y_1=0$ and $w=-1$ into the first inequality of his LP and get $y_3=1/3$, so his optimal strategy is $(0,2/3,1/3)$.


\newpage
\section*{Problem 5}
\stepcounter{problemnumber}
\begin{enumerate}[(a)]
\item Let $X=ax_1+bx_2+c$ and $Y=ay_1+by_2+c$. We algebraically manipulate the given inequality and see if it holds true:
\begin{align*}
\lambda f(x)+(1-\lambda)f(y) &\geq f(\lambda x+(1-\lambda)y) \\
\lambda f(\begin{bmatrix}x_1\\x_2\end{bmatrix})+(1-\lambda)f(\begin{bmatrix}y_1\\y_2\end{bmatrix}) &\geq f(\begin{bmatrix}\lambda x_1+(1-\lambda)y_1\\\lambda x_2+(1-\lambda)y_2\end{bmatrix}) \\
\lambda(ax_1+bx_2+c)^2+(1-\lambda)(ay_1+by_2+c)^2 &\geq (a(\lambda x_1+(1-\lambda)y_1)+b(\lambda x_2+(1-\lambda)y_2)+c)^2 \\
&= (\lambda(ax_1+bx_2+c)+(1-\lambda)(ay_1+by_2+c))^2 \\
\lambda X^2+(1-\lambda)Y^2 &\geq (\lambda X+(1-\lambda)Y)^2 \\
&= \lambda^2X^2+2\lambda(1-\lambda)XY+(1-\lambda)^2Y^2 \\
0 &\geq \lambda X^2(\lambda-1)+(1-\lambda)Y^2(1-\lambda-1)+2\lambda(1-\lambda)XY \\
&= \lambda(\lambda-1)(X^2+Y^2-2XY) \\
&= \lambda(\lambda-1)(X-Y)^2
\end{align*}
Looking at the last inequality, clearly $(X-Y)^2$ is positive, and because $0\leq\lambda\leq 1$ the other term $\lambda(\lambda-1)$ must be negative or zero, so the inequality holds and the objective function is convex.
\item The only three $(x_1,x_2)$ vertices are $(0,0)$, $(1,0)$ and $(0,1)$.
\end{enumerate}

\end{document}