\section{Lottery Scheduler}

\subsection{Overview}
The \texttt{Lottery Scheduler} extends the Priority Scheduler Class. It assigns every thread a number of "tickets" which determine the probability that it will get picked to have a resource. Shorter processes have higher priority. We solve the \texttt{Priority Inversion} problem by giving the running thread the summation of the tickets of all of the threads in queue for this resource.

\subsection{Correctness Constraints}

\begin{itemize}

\item The running thread has to have number of tickets equal to the summation (tickets of threads in waiting queue).
\item Tickets are not to be treated as objects, because we could potentially have billions of tickets in circulation at a time.
\item The number of tickets a process has is effectively the processes' priority. This means the the minimum priority if a process is 1 and the maximum priority is Integer.maxValue().
\item The total number of tickets in the system can not exceed Integer.maxValue().
\end{itemize}

\subsection{Declarations}

\textbf{LotteryScheduler extends PriorityScheduler:}\begin{itemize}

\item {\ttfamily priorityMinimum = 1}
\item {\ttfamily priorityMaximum = Integer.maxValue}
\item {\ttfamily newThreadQueue(transferPriority)} - Allocate a new lottery queue.
\item {\ttfamily getThreadState(thread)} - Return thread's scheduling state
\end{itemize}

\noindent\textbf{LotteryQueue extends PriorityQueue:} \begin{itemize}
\item {\ttfamily LThreadState holder} - Current holder. of this lock
\item {\ttfamily LinkedList<LThreadState> realQueue} - LinkedList holding all threads in line for this lock, I changed it from a priority queue, because there is no reason to compare threads now.
\item {\ttfamily LotteryQueue(transferPriority)} - calls super() then sets realQueue to a linked List instead of a Priority Queue.
\item {\ttfamily void waitForAccess(thread)} - puts a threads on this resource's wait queue.
\item {\ttfamily void acquire(thread)} - gives a thread the lock.
\item {\ttfamily KTread nextThread()} - returns the next thread what will get this resource and gives that thread the lock. Changed so that it works with LinkedLists and calls the correct functions in LThreadState.
\item {\ttfamily LThreadState pickNextThread()} - called by nextThread to choose the next thread that will get this lock. Changed so that it implements a lottery algorithm.
\item {\ttfamily int numTickets()} - Counts up the number of tickets allocated for this resource.
\end{itemize}

\noindent\textbf{LThreadState extends ThreadState:} \begin{itemize}
\item {\ttfamily LinkedList heldQueues}               - All the resources that this thread is holding.
\item {\ttfamily LotteryQueue waitQueue}              - A queue that this thread is waiting in.
\item {\ttfamily LThreadState(p)}                      
\item {\ttfamily void waitForAccess(waitQueue)}       - updates waitQueue.
\item {\ttfamily void release(heldQueue)}              - releases a lock and fixes priority for all threads affected.
\item {\ttfamily acquire(waitQueue)}                  - empties waitQueue and gives thread a lock.
\item {\ttfamily notifyPriorityChange()}              - called whenever priority of a thread is changed, updates priority for all threads connected to this one.
\item {\ttfamily recalculate()}                       - recalculates priority for current thread.
\end{itemize}

\subsection{Description}
\begin{verbatim}
class LotteryScheduler extends PriorityScheduler:
    int priorityMinimim
    int priorityMaximum
    
    newThreadQueue(transferPriority):
        return new LotteryQueue(transferPriority)
    
    
    getThreadState(KThread thread):
        return thread.schedulingState
    
    
    class LotteryQueue extends PriotityQueue:
        LThreadState holder
        LinkedList realQueue
        
        LotteryQueue(transferPriority):
            super()
            realQueue = LinkedList<LThreadState>()
        
        
        void waitForAccess(KThread thread):
            if transferPriority:
                holder.priority += thread.priority          
            threads.append(thread)                              
            thread.waitForAccess(this)                          
            
        
        LThreadState nextThread():
            oldHolder = holder                      
            oldHolder.priority = oldHolder.recalculate()
            holder = pickNextThread()
            remove new holder from queue
            holder.recalculate()
            notifyPriorityChange()
        
        void acquire():
            this.holder = getThreadState(thread)                
            this.holder.acquire(this)                           

        
        LThreadState pickNextThread():
            selection = Math.random from 1 to numTickets        
            for thread in threads:                               
                selection -= thread.priority                    
                if selection <= 0:                              
                    return thread           
        
        int numTickets():
            for all threads in queue:
                count up the number of tickets 
                    
        class LThreadState extends ThreadState:
            LinkedList heldQueue
            LotteryQueue waitQueue
            
            LThreadState(p):
                super()
                heldQueue = LinkedList
                
            waitForAccess(queue):
                this.waitQueue = queue
            
            acquire(queue):                                     
                this.waitQueue = null
                this.heldQueue.append(queue)
            
            release(heldQueue):                      
                heldQueue.remove(queue)
                this.priority = recalculate()
                notifyPriorityChange()
                
            notifyPriorityChange(value):
                if (waitQueue != null && waitQueue.holder != null && waitQueu.transferPriority):
            	 int a = waitQueue.holder.recalculate()
            	 waitQueue.holder.effPriority = a
            	 waitQueue.holder.notifyPriorityChange();
            
            int recalculate():
                for q in heldQueues:
                    effPriority += numTickets()
             
            
\end{verbatim}
\subsection{Testing Plan}
I will reuse many of the tests from proj1 PrioritySchduler.

\texttt{donationTest}
\begin{itemize}
\item R1 - holder(thread1 - 11), in queue: (thread2 - 2), (thread3 - 3)
\item priority of T1 - 6, T2 - 2, T3 - 3

\texttt{cascadeTest}
\item R1 - holder(T1 - 1), in queue (T2)
\item R2 - holder(T2 - 2), in queue (T3)
\item R3 - holder(T3 - 3)
\item T1's priority should be 6, T2's priority should be 5.
\item R2.nextThread(), now T1, T2, and T3's priority should be 3.
\end{itemize}
